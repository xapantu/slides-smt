\documentclass[]{article}
\usepackage{lmodern}
\usepackage{amssymb,amsmath}
\usepackage{ifxetex,ifluatex}
\usepackage[utf8]{inputenc}
\usepackage{fixltx2e}
\IfFileExists{upquote.sty}{\usepackage{upquote}}{}
\IfFileExists{microtype.sty}{%
\usepackage{microtype}
\UseMicrotypeSet[protrusion]{basicmath}
}{}
\usepackage{hyperref}
\hypersetup{unicode=true,
            pdftitle={Solving Counting Constraints on First Order Formulas in SMT},
            pdfborder={0 0 0},
            breaklinks=true}
\urlstyle{same}
\setlength{\emergencystretch}{3em}  % prevent overfull lines
\providecommand{\tightlist}{%
  \setlength{\itemsep}{0pt}\setlength{\parskip}{0pt}}
\setcounter{secnumdepth}{5}
\usepackage{tikz}
\usepackage{bussproofs}
\usepackage{syntax}
\usepackage{amsthm}
\usepackage{color}
\usepackage{algorithm}
\usepackage[noend]{algpseudocode}
\setcounter{secnumdepth}{3}

\title{Solving Counting Constraints on First Order Formulas in SMT}
\date{August 28, 2016}

\begin{document}
\maketitle

{
\setcounter{tocdepth}{3}
\tableofcontents
}

\newtheorem{definition}{Definition} \newtheorem{lemma}{Lemma}
\newtheorem{theorem}{Theorem} \newtheorem{property}{Property}
\newtheorem{example}{Example}

\newpage

\section*{Introduction}\label{introduction}
\addcontentsline{toc}{section}{Introduction}


This report describes work I did during an internship at SRI
International, in Menlo Park, CA. During this internship, I
worked on decision procedures related to symbolic model checking
of infinite, parametric systems, under the supervision of
Dejan Jovanović and Bruno Dutertre.

Model checking aims to establish whether a state-transition
system satisfies a property usually expressed in temporal logic.
An important special case is checking whether a state property
$P$ is an invariant of a system $S$, that is, whether $P$ is true in
all the reachable states of $S$. For infinite-state systems, this
problem can be solved using techniques such as bounded model
checking, $k$-induction, or IC3/PDR \cite{pdr}. All these techniques start with
a symbolic representation of the property $P$ and of the system $S$ in
first-order logic and rely on satisfiability modulo theories (SMT)
solvers as reasoning engines.

SRI's computer science laboratory is developing a model checker,
called \href{http://sri-csl.github.io/sally/}{Sally}, that implements bounded model checking, $k$-induction,
and a variant of IC3.  One of the main applications of Sally is the
verification of fault-tolerant protocols. Currently, Sally can
handle protocols with a fixed number of component, but it is being
extended to parametric protocols, where the number of components is
not fixed a priori. In this context, a key issue is to efficiently
handle the so-called counting constraints that occur frequently in
modeling fault-tolerant protocols. A typical example of counting
constraints is an assumption of the form ``less than one third of
the processes are faulty".

Several methods and tools have already been proposed to study parametric protocols. However, they
are not particularly suited to check fault tolerant protocols. For instance, \textsc{Cubicle}
\cite{ConchonGKMZ12} is suited to study a class of parameterized protocols including cache coherence
protocols, but is not equipped to deal with these counting constraints.

During my internship, I first implemented a decision procedure for counting
constraints in arithmetic then I extended it to counting constraints
with arrays. This decision procedure can then be used with Sally to check parametric fault-tolerant protocols. Both decision procedures are implemented as overlays to
an existing SMT solver. The decision procedure for arithmetic is
described in Section 1. Section 2 presents the extension of this decision procedure to arrays.

\ifthenelse{0=0}{}{

Model checking is used to prove properties of a system. A system is modeled by a set
of initial states and transitions. State transitions, initial states and properties are expressed in
first order logic. We are interested in infinite state systems. As it is not possible to enumerate
every state and verify they satisfy the properties, the goal is to find invariants of the system,
which are strengthening of the properties. Several techniques can be used to solve this problem,
such as bounded model checking, IC3 \cite{pdr}, or k-induction. For infinite state systems, model
checking is done symbolically and uses satisfiability modulo theories (SMT) solvers.

Extending this to support analysis of parametric systems is of major interest to prove correctness
of a large number of protocols, such as fault tolerant protocols\cite{AlbertiGP16} or cache coherence
protocols\cite{ConchonGKMZ12}. A key issue is handling efficiently cardinality constraints in
SMT.

During my internship, I worked on a decision procedure for counting constraints in arithmetic
(section 1), and then on an extension for counting constraints with arrays. Both decision
procedures were implemented as an overlay to an existing SMT solver.

I did my internship at SRI, California, under the supervision of Bruno Dutertre and Dejan
Jovanović.}

\section{Counting Constraints for Arithmetic}

We consider a set of distinct theories $\mathbf{T}_\mathbb{Z}$,
$\mathbf{T}_{eq}$, $\mathbf{T}_2$, \ldots{}, $\mathbf{T}_n$, where
$\mathbf{T}_\mathbb{Z}$ is the theory of $\mathbb{Z}$, and
$\mathbf{T}_{eq}$ is the theory of equality between the other theories.
Atoms and variables of these theories are defined as usual. In the
following, a first order formula without quantifiers is defined with the
grammar described in Figure \ref{formula}. Terms of the form
$c = \sharp\{ x\ |\ \psi(x) \}$ are called counting constraints. 
Term $\sharp\{ x\ |\ \psi(x)\}$ denotes the cardinality of a subset $S$
of $\mathbb{Z}$ such as $x \in S \iff \psi(x)$.

\begin{figure}[h]
\begin{grammar}
    
<formulas> $\phi$ ::= $\phi$ $\land$ $\phi$
\alt $\phi$ $\lor$ $\phi$
\alt $\lnot$ $\phi$
\alt <atoms> of $\mathbf{T}_i$
\alt $c = \sharp\{ x \mid y\ \text{\textless}\ x \le z \land \phi(x) \}$ where $c, y, z$ are variables of $\mathbf{T}_\mathbb{Z}$

\end{grammar}

\caption{Formula syntax}
\label{formula}
\end{figure}

\begin{example}
The formula $c = \sharp\{x \mid 0 < x \le 10 \land 5 < x \le 10\}$ is equivalent to the formula $c = 5$.
\end{example}

\begin{example} The formula $c = \sharp\{x \mid 0 < x \le c + 1\} \land c > 0$ is equivalent to the formula $c = c+1 \land c >0$ and is not satisfiable.
\end{example}

\begin{example} Let  $F \equiv c = \sharp\{x \mid 0 < x \le y \land 0 < x \le z\}$. In this case, $c$ can
not be directly expressed with an arithmetic expression, as the symbolic value depends on whether $y < z$. However, with a disjunction on $y >z$, $F$ is also equivalent to a pure arithmetic formula $(y > z \implies c = y) \lor (y \le z \implies c = z)$.
\end{example}

Counting constraints provide a way to specify how many integers satisfy
a given predicate. Thus, they generalize quantifiers over the
integers. An existential
quantifier can be transformed into a constraint involving a cardinality
greater than 1, and a universal quantifier into a counting constraint
composed of a cardinality equal to 0 (in this case, the predicate will
be the negation of the formula).

Dealing with quantifiers in SMT formula is known to be a hard problem.
Some theories, such as arithmetic, support quantifier elimination \cite{cooper}, but
quantifier elimination is expensive and is not available in every theory \cite{bradley2006s}. Even when quantifier elimination algorithms exist, simpler
methods based on variable instantiation may work better in practice \cite{dutertre2015solving}. 

That's why we look at a restricted subset of formulas built on the syntax of Figure \ref{formula}:
there is no nested counting constraints. Furthermore, to keep the algorithm simple, if the
quantified variable is $x$, every atom of the counting constraint where $x$ appears must be of the form $x < t$ or $t \le x$,
where $t$ is an arithmetic term where $x$ does not appear. Without loss of generality, we only
consider the case where $t$  is an arithmetic variable (a new variable equal to an arbitrary term
can always be added to the formula). Atoms such as $2x < y$ are not handled by our algorithm.
Handling them can be done using divisibility constraints, but this cause extra complexity.

It can be shown that every formula in this class is equivalent to a formula of the form:
\begin{equation}
F(c_1, \ldots, c_n, \mathbf{y}) \land \bigwedge\limits_{i=1}^n c_i = \sharp\{ x \ |\ \psi_i(x, c_1, \ldots, c_n,
\mathbf{y})\}
\label{maingoal}
\end{equation}
where no counting constraints appear in $F$.

\subsection{Decision Procedure}


We now present an algorithm that decides whether a formula of the form
\ref{maingoal} is satisfiable and if so produces a model. 

\subsubsection{Overview}

We assume that we have an SMT solver $\mathcal{S}$ that decides satisfiability
of formulas in theories $\mathbf{T}_\mathcal{Z}, \mathbf{T}_{eq}, \mathbf{T}_1, \ldots,
\mathbf{T}_n$ (i.e. formulas
without counting constraints). We also assume that $\mathcal{S}$ can produce
models of satisfiable formulas. 

Our decision procedure translates every counting constraints to arithmetic expressions. To do that, it needs
to make additional assumptions (example 3). We use models constructed by the SMT solver to enumerate every possible set of
assumptions compatible with formula $F$.

Our procedure works as follows: first, the formula without the counting constraints is forwarded to
the solver we already have. If it is \texttt{unsat}, then the formula with the counting constraints is also
\texttt{unsat}. Otherwise, we get a model and we compute a symbolic arithmetic expression for every counting
constraints, under the assumptions that hold in the model. Then, both the assumptions and the value
of the counting constraints are added as new clauses to the solver. If $\mathcal{S}$ answers \texttt{sat}, then we
have a model which satisfies the formula with the counting constraints. Otherwise, we restore the
solver to its previous state, we assert the negation of the assumptions and we
retry the procedure.

A pseudo-code description is shown in algorithm \ref{arith}.

\begin{algorithm}[h]
\caption{Satisfiability of arithmetic formula with counting constraints}\label{arith}
\begin{algorithmic}[1]
\State \Call{assert}{$F(\mathbf{y}, c_1, \ldots, c_n)$}
\While{$\mathcal{M} = $ \Call{check-sat}{\ } }
    \State \Call{push}{\ }
    \State $A \gets \emptyset$
    \ForAll{ $i$ in $[1..n]$}
        \State $A_i, S_i \gets $ \Call{interpret-constraint}{$c_i$, $\phi_i$,
        $\mathcal{M}$}
        \State $A \gets A \cup A_i$
        \If{$S_i$ is infinite}
            \State \Call{assert}{$\lnot A$}
            \State \Call{continue}{}
        \EndIf
    \EndFor
    \State \Call{assert}{$A$}
    \State \textsc{assert}$\left(\bigwedge\limits_{i=1}^n c_i = \sum\limits_{[a, b] \in S_i} b - a\right)$
    \If {\Call{check-sat}{\ } }
        \State \Call{pop}{\ }
        \State \Return{sat}
    \EndIf
    \State \Call{pop}{\ }
    \State \Call{assert}{$\lnot A$}
\EndWhile
\State \Return{unsat}
\end{algorithmic}
\label{arith}
\end{algorithm}

We assume that our SMT solver $\mathcal{S}$ supports the following operations (besides the variable definitions):
\texttt{assert} (adds a formula to the current context),
\texttt{check-sat} (checks the satisfiability of the conjunction of the
formulas in the current context), \texttt{push} (creates a new context
with the current context formula), \texttt{pop} (restores the context to
the last \texttt{push} call). These operations are supported by most
modern SMT solvers which can work in an incremental way.


\begin{example}
\ \newline Let $F \equiv N > z \land N > u \land z \ge 10 \land c > z \land u \ge 0 \land c = \sharp\{ x \mid 0 \le x < N \land (x < z \lor x < u) \}$.
For this formula, the algorithm works as follows:
\begin{itemize}
\item \textsc{assert}$\left(N > z \land N > u \land z \ge 10 \land c > z \land u \ge 0 \right)$.
\item \textsc{push}$()$
\begin{itemize}
	\item First, get a model: $\mathcal{M} = \{ z \mapsto 10, N \mapsto 11, u \mapsto 0, c \mapsto 11\}$.
	\item Compute the domain of $\{ x \mid 0 \le x < N \land (x < z \lor x < u) \}$.
	\item The assumptions are $N > z, u < z, N > 0, N > u$ and the domain is $S = \{ [0, z) \}$.
	\item \textsc{assert}$\left(N > z \land u < z \land N > 0 \land N > u\right)$
	\item \textsc{assert}$\left(c = z)\right)$
	\item Then it is \texttt{unsat}, as we can't have both $c = z$ and $c> z$
\end{itemize}
\item \textsc{pop}()
\item The assumptions were not correct, we want to try something else.
\item \textsc{assert}$\left(\lnot \left(N > z \land u < z \land N > 0 \land N > u\right)\right)$
\item \textsc{push}$()$
\begin{itemize}
\item Get a new model: $\mathcal{M} = \{ z \mapsto 10, N \mapsto 30, u \mapsto 20, c \mapsto 11\}$.
\item Compute the domain and assumptions, it gets $A = \{ N \ge u, u > z, N > 0, u > 0\}$ and $S = [0, u)$.
\item \textsc{assert}$\left(N \ge u \land u > z \land N > 0 \land u > 0\right)$
\item \textsc{assert}$\left(c = u\right)$.
\item Now it is \texttt{sat}, and the model $\mathcal{M} = \{ z \mapsto 10, N \mapsto 30, u \mapsto 20, c \mapsto 20\}$ satisfies the whole formula.
\end{itemize}
\end{itemize}

\end{example}

\subsubsection{Constraints Interpretation}

In this section, we explain how a set $\sharp\{x\mid\phi(x)\}$ can be symbolically computed under
some assumptions.

\begin{definition}[Assumptions]

We call assumptions a set $A$ whose elements are literals of the theories
$\mathbf{T}_i$. In the context of a first order formula, writing $A$
means the conjunctions of the atoms of $A$.

\label{assumptions}

\end{definition}


\begin{definition}[Symbolic Interval]

A symbolic interval is a couple of values that are either arithmetic
variables, constants or $\infty$. If $I = [a, b)$, $x \in I$ is
defined as $x < b \land a \le x$. \newline~\newline
It is finite if none of the bounds are infinite.

\label{symbolic}

\end{definition}


\begin{definition}[Arithmetic Domain]

An arithmetic domain is a finite set of symbolic intervals (definition
\ref{symbolic}). It is  associated to some assumptions
(definition \ref{assumptions}) which make them disjoint and ensure that
the lower bound of an interval is lower than the upper bound.

\label{arithmetic}

\end{definition}

Let S be an arithmetic domain. We write $x \in S$ for
$\left(\bigvee\limits_{I \in S} x \in I\right)$.

Let $\psi_i$ be one of the formula of the counting constraints. We are going to give rules to compute $(S, A)$ such as
$A \implies \forall x \left(\psi_i(x, \mathbf{y}) \iff x \in S\right)$. The goal is to express the
cardinality $\sharp\{x \mid \psi_i(x)\}$ with an arithmetic formula (see below), which is easy if $S$ is only
comprised of distinct intervals. That is why we want the set $S$
to satisfy the following property:


\begin{property}[Distincts]

Let $S$ be an arithmetic domain computed with the rules described above, and $A$ the associated set of assumptions.
Let $I, J \in S$ such that $I \neq J$.

Then, $A \implies (x \in I \implies x \not\in J)$, i.e.
$\{x\ |\ x \in I\}$ and $\{x\ |\ x \in J\}$ are
disjoints\footnote{Here, $x \in I$ is still defined as $a \le x \land x < b$ if $I = [a, b)$.}.

\label{distincts}

\end{property}


The arithmetic domain and the associated set of assumptions is computed inductively on the formula
$\psi_i$.

\paragraph{Base cases}

If the formula is $y \le x$ where $y$ is an arithmetic variable, then the arithmetic domain is
$\{[y, +\infty)\}$ with no assumptions.

If the formula is $x < y$, then the arithmetic domain is $\{(-\infty, y)\}$, with no assumptions.

If the formula is a literal $l$ of a theory $\mathbf{T}_i$, then we look at the value of $l$ in
$\mathcal{M}$. If it is \texttt{true}, then $S = \{(-\infty, +\infty)\}$ and the assumptions are
$\{l\}$. If it is \texttt{false}, then $S = \emptyset$ and the assumptions are $\{\lnot l\}$.

Property \ref{distincts} clearly holds.

\paragraph{Conjunction}

Let $S$ and $S'$ be two arithmetic domains computed for $\phi$ and $\psi$, and $A$, $A'$ the sets of assumptions associated to
them. The domain of $\phi\land\psi$ is the intersection of $S$ and $S'$. To construct this intersection $S \sqcap S'$, we are going to
do the intersection of an interval $I$ of $S$ and an interval $J$
of $S'$.

So, let $I = [a, b)$ and $J = [c, d)$ be two symbolic intervals. We want a new
interval $K_{I, J} = [e, f)$ such as
$A \cup A' \cup A_{S \sqcap S'} \implies \left(x \in I \land x \in J \iff x \in K_{I, J}\right)$.
$A_{S \sqcap S'}$ is a set of additional assumptions that are added to make it possible to define
the
intersection. For instance, if the two intervals are $[0, x)$ and $[0, y)$, then $x < y$ or $y \le x$
must be added to claim that the intersection is either $[0, x)$ or $[0, y)$.

To choose what are the assumptions needed to compute the intersection, we are going to use the model
$\mathcal{M}$
of the formula $F$, so as the consistency between the assumptions is guaranteed. For any arithmetic variable $x$, $x_\mathcal{M}$ is
defined as the value of $x$ in this model, and more generally for any
term $t$ of theory $T_{\mathbb{Z}}$ (and $+\infty$, $-\infty$),
$t_{\mathcal{M}}$ is the interpretation of $t$ in this model.

%In what follows, we describe a way to compute $K_{I, j}$ as well as
%$A_{S \sqcap S'}$ in regards to the model, so as the assumptions are
%consistent (and the model satisfies them).

If
$max(a_\mathcal{M}, c_\mathcal{M}) < min(b_\mathcal{M}, d_\mathcal{M})$,
then $K_{I, J} = [max(a, c), min(b, d))$ (where $max(a, c)$ is $a$
if $a_\mathcal{M} > c_\mathcal{M}$, else $c$), else it is undefined
(because in the model, the interpretation of the intervals are indeed
disjoints).

The intersection of $I$ and $J$ is guided by the model.
Every time there is a decision to take (such as $a < c$), the values
of the model are looked at. Those decisions are recorded, as they
are the assumptions required for this interpretation to be correct. The
set $A_{S\sqcap S'}$ is composed of those decision. By induction, it is clear that if the same model
$\mathcal{M}$ is used for every intersection, then $\mathcal{M}$ satisfies $A \cup A' \cup A_{S \sqcap S'}$.

%It is also clear that by induction, if $S$ and $S'$ have the
%property \ref{distinct}, $S\sqcap S'$ also respect this property.

Then, the intersection of the domains is defined as
$S \sqcap S' = \{ K_{I, J} \ |\ I \in S, J \in S' \}$. Some intervals $K_{I, J}$ may not be defined
if they are empty. The associated assumptions set is $A \cup A' \cup A_{S \sqcap S'}$.

Property \ref{distincts} clearly holds if it holds for both $S$ and $S'$.

\begin{example}

Let $S = \{ I_1, I_2\} =  \{[0, 10), [12, z)\}$, $A = \{ z > 12 \}$, $S' = \{J\}= \{[15, y)\}$, $A =
\{ y > 15 \}$.

The model $\mathcal{M}$ is $\{ z \mapsto 16, y \mapsto 20 \}$.

The algorithm intersects $I_1$ with $J$,  which is $\emptyset$ as $12 > 10$. Then it intersects
$I_2$ with $J$. As $max(15, 12) = 15 < min(z_\mathcal{M}, y_\mathcal{M}) = 16 = z_\mathcal{M}$, the
intersection is $[15, z)$. The assumptions needed to do this intersection are $15 > 12, y > z, z >
15$.

The arithmetic domain is then $S\sqcap S' = \{[15, z)\}$ with the associated assumptions $\{z > 12,
y > 15, 15 > 12, y > z, z > 15\}$.

\end{example}

\paragraph{Negation}
We want to compute an arithmetic domain for a formula $\lnot \phi$. We assume we have an arithmetic
domain $S$ (with assumptions $A$) computed for $\phi$. Then property \ref{distincts} holds. There
are several cases, but here I only explain how the algorithm works if there is no $\infty$ in $S$.

By looking at the values in the model $\mathcal{M}$ of the bounds of the intervals in $S$, we can
choose inequalities that make $S = \{[a_1, b_1), \ldots, [a_k, b_k)\}$ ordered, i.e. $\forall i$, $A
\implies b_i < a_{i+1}$.

Then, the arithmetic domain for $\lnot \phi$ is $S' = \{(-\infty, a_1), [b_1, a_1), \ldots, [b_k,
+\infty)\}$, and the assumptions are those in $A$ and the new ones chosen to order $S$.

%\begin{definition}[Domains intersection, $S \sqcap S'$ and $A_{S \sqcap S'}$]

%If $S$ and $S'$ are two arithmetic domains associated to the
%assumptions $A$ and $A'$, then the intersection $S\sqcap S'$ is
%defined as an arithmetic domain such as\newline
%$A \cup A' \cup A_{S\sqcap S'} \implies \left(\left(x \in S \land x \in S'\right)\iff x \in S \sqcap S'\right)$\newline (where
%$x \in S$ means $\left(\bigwedge\limits_{I\in S} x \in I\right)$).

%\label{domains}

%\end{definition}

%\vspace{3mm}

%\begin{definition}[Complementary Domains, $S^c$]

%If $S$ is an arithmetic domain, then $S^c$ is defined as a set of
%intervals such as
%$\forall x.\ \left( \exists I \in S.\ x \in I\right) \iff \left( \not\exists I \in S^c.\ x \in I\right)$.

%\label{complementary}

%\end{definition}

%\vspace{3mm}


\begin{lemma}[Correctness]

Let $S$ be an arithmetic domain computed with the rules described above, and $A$ the associated set of assumptions.

Then,
$A \Rightarrow \left(\phi(x, \mathbf{y}) \iff x \in S\right)$.

\label{correctness-interpretation}

\end{lemma}

\begin{proof}
The lemma clearly holds for the base cases. It is correct by construction for the conjunction and the negation as there is an immediate correspondence with intersection and complementary domains.
Thus, the lemma holds by induction.
\end{proof}




A corollary of property \ref{distincts} is the fact that the cardinality of the reunion of the elements
of $S$ is the sum of the cardinality of its elements:
\begin{equation}
A \implies \sharp\{ x \in S \} = \sum\limits_{I \in S} \sharp\{x \in I\}
\end{equation}

Moreover, as the formula inside every counting constraints is of the form $t_1 \le x < t_2 \land
\phi$ (Figure \ref{formula}), it is clear that every interval in S is finite. Then, if $S$ and $A$
were computed for a formula $\phi$:
\begin{equation}
A \implies \sharp\{x \mid \phi(x, \mathbf{y})\} = \sum\limits_{[a, b)\, \in\, S} b - a
\end{equation}


\subsubsection{Termination and Correctness}

\begin{lemma}[Termination]

Algorithm \ref{arith} terminates.

\label{termination}

\end{lemma}

\begin{proof}
In the former section, we explained how the assumptions set was computed. An
assumption can be an equality, a disequality or an inequality between two terms
that appear in the formulas $\phi_i$. Thus, there is a finite number of possible
set of assumptions.

At every iteration, there is a $\text{\textsc{assert}}(\lnot A)$, hence the fact that there is
a finite number of iterations.
\end{proof}

\begin{lemma}[Correctness]

Algorithm \ref{arith} is correct.
\end{lemma}

\begin{proof}
Let $G = F \land \bigwedge\limits_{i} c_i = \sharp\{x\mid\phi(x)\}$ be a formula.

If algorithm \ref{arith} returns a model, then with lemma \ref{correctness-interpretation} and property
\ref{distincts} we can conclude that the model satisfies the formula, as every counting variable of
$F$ is set to the correct arithmetic expression.

If it returns \texttt{unsat}, and a model $\mathcal{M}$ satisfies $F$ nonetheless. Then $\mathcal{M}$
satisfies a set of assumptions which has been ruled out by the algorithm (else, as $\mathcal{M}
\vdash F$, it would be sent by the SMT solver), but the variables $c_i$ do not have a compatible
interpretation in $\mathcal{M}$ with the symbolic expression computed by the algorithm. Thus, it
contradicts lemma \ref{correctness-interpretation} or property \ref{distincts}.
\end{proof}

\subsection{Implementation}

I implemented the algorithm described above as an overlay on an existing SMT solver. Any solver
compatible with the SMT-LIB format \cite{barrett2010smt} and which can work in incremental mode can
be used.I mostly experimented with Yices and Z3. The code is available at
\url{https://github.com/xapantu/counting-smt/}.

The counting solver takes a file in the SMT-LIB format \cite{barrett2010smt}, with some extensions
to specify the constraints. It forwards most of the commands to the underlying solver. It only
extracts the syntax extension to specify counting constraints and execute the algorithm \ref{arith}
when it sees $\texttt{check-sat}$.

The counting solver and the SMT solver interacts via plain text SMT-LIB  commands, which is far from
optimal. The obvious benefits are that it is easier to implement and most SMT solvers are compatible
for free.


\ifthenelse{0=0}
{
\section{Counting Constraints with
Arrays}

In this section, we describe an extension of the previous algorithm to
deal with arrays. The syntax for $k$ arrays $a_1, \ldots, a_k$ is
described in Figure \ref{syntaxarray}. It is important to note that
arrays are only accessed on the quantified variable, and not on a
general term built on this variable (such as $x + 1$, or a nested
array read). Removing this syntax restriction leads to an undecidable
array theory fragment, as stated in
\cite{bradley2006s}, even for small additions to the
fragment.

In additions to the restriction already described in the first section, it is also necessary for the
formulas inside counting constraints to not have atoms with both an array read $a_i[x]$ and $x$,
that is, atoms like $a_i[x] = x$ are forbidden.

\begin{figure}[h]
\begin{grammar}
    
<formulas> $\phi$ ::= $\phi$ $\land$ $\phi$
\alt $\phi$ $\lor$ $\phi$
\alt $\lnot$ $\phi$
\alt $\langle atoms\rangle$ of $\mathbf{T}_i$
\alt $c = \sharp\{ x \mid y\ \text{\textless}\ x \le z \land \phi(x, a_1[x], \ldots, a_k[x]) \}$ where $c, y, z$ are variables of $\mathbf{T}_\mathbb{Z}$

\end{grammar}
\caption{Array Extension}
\label{syntaxarray}
\end{figure}


An array has a size, which is an arithmetic variable of the theory
$\mathbf{T}_\mathbb{Z}$. This is similar to the model checker \textsc{Arca}
\cite{AlbertiGP16} (with the subtle difference that
different arrays can have different size) but unlike \textsc{Cubicle}
\cite{ConchonGKMZ12}, whose fragment does not
provide a syntax to express an array length. In the context of fault
tolerant systems, this is an important detail, as we typically want to
specify that at most one third of the components are faulty.

The elements of the arrays can be of any sort defined in a theory $\mathbf{T}_i$.

As soon as there is an array term in the counting constraints, the
cardinality can no longer be infinite, as the array can only be accessed
on a finite interval, that is, term $a[x]$ implicitly implies that $0 \le x < N$ if $a$ is an array of size $N$.

It may be interesting to have array reads outside of the counting
constraints. However, they can be rewritten as counting constraints, as in
\cite{bradley2006s} or
\cite{AlbertiGP16}: $a[v]$ can be replaced by a new variable $y$ with the additional constraint:
\begin{equation}
\sharp\{x\mid x = v \land a[x] = y\} = 1
\end{equation}

In the implementation section, we explain
how this algorithm can be changed to work with the usual theory of arrays of an
SMT solver, which is more efficient.

\subsection{Decision Procedure}

\subsubsection{Overview}

The algorithm to solve counting constraints with arrays is mostly the
same as the one to solve pure arithmetic counting constraints. While we do the interval analysis,
we must save the constraints on the arrays. In the end, those constraints must be checked for
satisfiability and whether they conflict in case there are several constraints for the same interval.

During my internship, I experimented several possible algorithms to
manipulate those constraints. The algorithm I describe here might seem a
bit brutal as it extensively rely on the SMT solver, but it
worked better than the other attempts, probably because a modern SMT
solver can be much more efficient than a less optimized specialized
algorithm.

\begin{algorithm}[h]
\caption{Satisfiability of arithmetic and formula with counting constraints}
\begin{algorithmic}[1]
\State \Call{assert}{$F(\mathbf{y}, c_1, \ldots, c_n)$}
\While{$\mathcal{M} = $ \Call{check-sat}{\ } }
    \State \Call{push}{\ }
    \State $A \gets \emptyset$
    \ForAll{ $i$ in $[1..n]$}
        \State $A_i, S_i \gets $ \Call{interpret-constraint}{$c_i$, $\phi_i$,
        $\mathcal{M}$}
        \State $A \gets A \cup A_i$
        \If{$S_i$ is infinite}
            \State \Call{assert}{$\lnot \left( A \right)$}
            \State \Call{continue}{}
        \EndIf
    \EndFor
    \State \Call{slice}{$(S_i)_i$}
    \State \Call{partition}{$(S_i)_i$}
    \State \Call{assert}{$A$}
    \State \Comment{$v_\alpha$ is a variable corresponding to a constraint associated to $S_i$}
    \State \Call{assert}{$\bigwedge\limits_{i=1}^n c_i = \sum\limits_{[a, b] \in S_i} \sum\limits_{\alpha} v_{\alpha}$}
    \State \Call{assert-constraints}{$(S_i)_i$}
    \If {\Call{check-sat}{\ } }
        \State \Call{pop}{\ }
        \State \Return{sat}
    \EndIf
    \State \Call{pop}{\ }
    \State \Call{assert}{$\lnot \left( A \right)$}
\EndWhile
\State \Return{unsat}
\end{algorithmic}
\label{arrayalgo}
\end{algorithm}

\begin{example}
\end{example}

\subsubsection{Constraints on the Arrays}

\begin{definition}[Array Constraint]

An array constraint is a first order, quantifier free, formula whose
free variables are the free variables of the formula and the variables
$a_1[\cdot], \ldots, a_k[\cdot]$.

\label{array}

\end{definition}


\begin{definition}[Domain]

A domain is a finite set of symbolic intervals (definition
\ref{symbolic}), every one of them associated to an array constraint
(definition \ref{array}).

\label{domain}

\end{definition}

An arithmetic domain (definition \ref{arithmetic}) can then be seen as a
domain whose every symbolic interval is associated to the constraint
\texttt{true}.

Generating an array constraint for every symbolic interval is done while we compute those symbolic
intervals. The rules for conjunction and negation are updated accordingly. For instance, when we
look at a conjunction and produce the intersection of every symbolic interval in both domains, we
associate to this intersection the conjunction of the array constraints of both intervals.

There are additional base cases, which are the literals with $a_1[x], \ldots, a_k[x]$. For those
case, we return the domain with one symbolic interval, $(-\infty, +\infty)$, associated to the
constraint which is the literal, and to an empty assumptions set.

\subsubsection{Checking the Constraints on the Arrays}

\ifthenelse{0=0}{}{
We now want to generate a set of constraints that are equivalent to the
satisfiability of the formula and the counting constraints.

We start from a set of $l$ domains (one for every counting
constraints) and generate a set of arithmetic constraints.

\paragraph{Slice}\label{slice}

The first operation we do is what we call \emph{slicing}. It means that
the domains must be transformed so as they all have the same intervals.
In practice, we are looking for a subdivision of $[A, B)$ (where $A$
is the lower bound of every array index and $B$ the upper bound) which
can be used to express every domains. This is a simple operation that I
do not detail here, intuitively every bound of the intervals is
collected, then they are split into equality classes and ordered (in the
meantime the set of assumptions may be made bigger).

\paragraph{Partition}\label{partition}

At this point we have a set of domains who all have the same intervals
but different array constraints associated to them. For an interval
$I$, we consider every constraints associated to it, i.e.
$\phi_1, \ldots, \phi_l$. From these one we can create a set of formula
$\psi_1, \ldots, \psi_{l'}$ which are a
partition\footnote{A partition is a set of formula $\psi_1, \ldots, \psi_n$ such as $\psi_1 \lor \ldots \lor \psi_n \equiv \top$ but $\forall i, j. \; \psi_i \land \psi_j \equiv \bot$.}.
There can be at most $2^l$ formula, but of course that can be
dramatically reduced in practice with heuristics such as memoization are
inclusion detection.

\paragraph{Arithmetic Constraints}\label{arithmetic-constraints}

$I = [a, b)$ is supposed to be a finite interval (or it means that the
array is accessed on an infinite intervals, which is not supposed to
happen), so, it has a length $b - a$. As $\psi_1, \ldots, \psi_{l'}$ is a
partition, it holds that
$b - a = \sum\limits_{i = 1}^{l'} \sharp\{x \ |\ \psi_i(a_1[x], \ldots, a_n[x])\}$.We
create a new variable $v_i$ (such as $0 \le v_i$) for every
$\sharp\{x\ |\ \psi_i(a_1[x], \ldots, a_n[x]\}$ and adds the constraint
$b - a = \sum v_i$ to the solver. As long as every $\psi_i$ is
satisfiable, then we can build arrays that satisfy
$v_i = \sharp\{x\ |\ \psi_i(a_1[x], \ldots, a_n[x]\}$. Hence, we must add
$v_i > 0 \implies \psi_i(a_{i, 1}, \ldots, a_{i, n})$ (with new variables
$a_{i, 1}, \ldots$.

\paragraph{Algorithm}\label{algorithm-1}

See algorithm \ref{arrayalgo}.

\paragraph{Example}\label{example-1}

TODO
}

\subsection{Implementation}

\paragraph{Theory of arrays} The previous algorithm transforms every array read and write into an
array constraint. This is not very efficient, because the underlying SMT
solver has no clue regarding the consistency of the values it provides
for an array read. That's why it is interesting to change this algorithm
to make it work with a usual array theory of an SMT solver. The
algorithm then has to take into account the decision that the SMT solver
when it outputs the constraints, so as the result is consistent
(i.e.~the arrays are not defined twice).

Unfortunately, as our algorithm operates outside of the solver, the
interactions they can have are quite limited. Thus, it is important to
understand how the theory of arrays is usually implemented to make it
work with the counting constraints algorithm described in the last
section.

A state of the art implementation of the theory of arrays is described
in \cite{de2009generalized}. More specifically, it
explains how works the SMT solver Z3.

The only two operations on arrays are \texttt{select} and
\texttt{store}. \texttt{(select\ a\ x)} access the array $a$ at index
$x$ (also written $a[x]$), while \texttt{(store\ a\ x\ b)} creates a
new array whose elements are the same as $a$ but for $x$, where it
is set to $b$. The two axioms of this theory are:

\begin{subequations}
    \begin{align}
        \forall a:(\sigma \implies \tau), i:\sigma, v:\tau\, .\, store(a, i, v)[i] = v
        \\
        \forall a:(\sigma \implies \tau), i:\sigma, j:\sigma\, .\, i \neq j \implies store(a, i, v)[j] = a[j]
    \end{align}


    Additionnaly, there is the extensionality axiom:

    \begin{align}
        \forall a:(\sigma \implies \tau), b:(\sigma \implies \tau)\, .\, a \neq b \iff \exists i\: a[i] \implies a[i] \neq b[i]
    \end{align}
\end{subequations}

For the first two axioms, it is clear that the SMT solver is only going
to take decisions about terms index which appear inside a \texttt{store}
and a \texttt{select}. An equality might introduce additional indexes to
make two arrays different. That's why an array extracted from a model of
an SMT solver always follows the same pattern: special values are
defined for a finite number of terms (an over-approximation of those
terms can be obtained by looking at the formula), and then a default
value. And the default value does not matter, as long as there is no
equality.

\begin{example}
\end{example}


Thus, to ensure the consistency of the counting constraint algorithm and
the SMT solver decisions about arrays, we only have to take into account
the value at those index terms, and array equality if needed.

The first problem can be solved by tracking those index terms, then
looking at the model which slice they belong to, and group them by
slice. Then, for every array constraint $\phi_\alpha$ of this slice,
some new clauses must be sent to the solver to ensure the consistency of
this values decided by the SMT solver with the number of $x$ which
satisfies $\phi_\alpha$

The equality between two arrays must also be tracked, they must be taken
into account when the arithmetic constraints are expressed at the end of
the algorithm by adding new array constraints. In my implementation, some heuristics (such as
substitution of the arrays when there is equality) are also used.

However, for model checking, one way of the extensionality axiom can be dropped. If $a \neq b$, it
does not have to reflect in the existence of an $i$ such as $a[i] \neq b[i]$. In our example, an
equality is always produced by an assignment, and that is why $\lnot (a = b)$ is better interpreted
as an undefined behavior than a disequality. My tests showed that making equality not extensional
(and thus avoiding an extra array constraint) did not provide a significant performance gain, that
is why my implementation has an extensional equality, as it would be confusing for no gain.


\paragraph{Complexity}

Our algorithm works by assuming a set of assumptions and checking that
cardinality values that they imply is consistent. The worst case for the
assumptions is to be a total ordering over every integer variable and an
assumptions on every atom of the formula. The number of total ordering
over $n$ variables is
$n!$\footnote{More than that when the case where two variables are equal is taken into account.},
which makes the algorithm unpractical if there is not enough constraints
on those variables in the formula.

That is why it would be interesting to have a better explanation of why
it is not \texttt{sat} and not only learn that the formula is
\texttt{unsat} with this particular ordering. That requires an
explanation from the SMT solver and a better integration with both the
arithmetic theory and the arrays theory.

Furthermore, partial assumptions are sufficient to compute the bounds of
a cardinality, and may also be sufficient to detect an inconsistency
earlier in the solving of the formula, hence avoiding useless
computations. Some work for abstract sets has already be done in this
direction \cite{cardinalityset}.

The way the algorithm works is not very different from the way a theory
is usually modeled in DPLL(T) based solver, the main differences being
that it has nested formula and has to interact with the array and
arithmetic theories. So, adapting it to be just another theory in a
solver may be worthwhile.

\ifthenelse{0=0}{}{
\newpage

\section*{Conclusion}

related work
optimization stuff

In \cite{AlbertiGP16, schweikardt}, there is already a way to solve
those formula. However, while they definitively explain that the problem
is decidable and why, they heavily rely on quantifier elimination and
thus may not be very practical nor easily integrated in modern SMT
solvers.
}
}{}

\newpage
\bibliographystyle{plain}
\bibliography{report}

\end{document}
