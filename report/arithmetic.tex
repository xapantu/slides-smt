% !TEX root = report.tex
% !TEX program = pdflatex

We will work in the setting of satisfiability modulo theories
\cite{barrett2009satisfiability}, specifically in the quantifier-free
theory of linear integer arithmetic $\qflia$. As usual, we assume a
set of variables ranging over $\mathbb{Z}$, and atoms as linear
constraints over these variables. A model $\mathcal{M}$ in $\qflia$ is
an assignment of variable to values in $\mathbb{Z}$. Given a model
$\mathcal{M}$, we can construct a new model where we assign a variable
$x$ to a value $v$ and the rest of the assignment unmodified, which we
denote with $\mathcal{M}^v_x$. As usual, when a formula $\phi$
evaluates to true in the model $\mathcal{M}$, we say that $\phi$ is
satisfied in $\mathcal{M}$ and denote this with $\mathcal{M} \vDash
\phi$.

We extend the usual language of $\qflia$ with counting constraints and
call this theory $\qfclia$, with formulas in this theory described in
Figure \ref{formula}.

\begin{figure}[h]
\begin{grammar}
<formulas> $\phi$ ::= $\phi$ $\land$ $\phi$
\alt $\phi$ $\lor$ $\phi$
\alt $\lnot$ $\phi$
\alt <atoms> of $\qflia$
\alt $c = \bcterm{l}{u}{\phi}$
\end{grammar}

\caption{Formula syntax}
\label{formula}
\end{figure}

We call the terms of the form $\bcterm{l}{u}{\phi}$ \emph{counting
terms}, and formulas of the form $c = \bcterm{l}{u}{\phi}$
\emph{counting constraints}. Counting terms denote the number of
solutions for $k$ in the interval $[l, u]$ that satisfy the formula
$\phi$\todo{Can $\phi$ include other counting constraints?}. More
formally, given a model $\mathcal{M}$ that assigns all variables in
$l, u, \phi$, the term $\bcterm{l}{u}{\psi}$ denotes the cardinality of
the set $S \subseteq \mathbb{Z}$ such that
\begin{align*}
  v \in S \iff v \in [l, u] \land \mathcal{M}^v_k \vDash \phi\enspace.
\end{align*}
Since all terms of the counting term, other than $k$, evaluate in
$\mathcal{M}$, the possible range of values that $k$ can take is
always bounded, ensuring that semantics are well defined.

\begin{example}
\label{ex:semantics}
Consider the formula
\begin{align*}
  c = \bcterm{0}{10}{5 < k \le 10}\enspace.
\end{align*}
This formula is satisfied in a model that assignes $c \mapsto 5$.
\end{example}

\begin{example}
\label{ex:simple}
The formula $c = \cterm{0 < k \le c + 1} \land c > 0$ is
not satisfiable. This is because the formula is equivalent to the
formula $c = c+1 \land c >0$ which is not satisfiable.
\end{example}

Example above illustrates that, in the theory $\qfclia$, formulas with
counting constraints can be reduced to formulas in $\qflia$. But, in
order to do so, in general many different cases need to be considered.

\begin{example}
\label{ex:disjunction}
Let  $F$ be the formula
\begin{align*}
  c = \cterm{0 < k \le x \land 0 < k \le y}\enspace.
\end{align*}
In this case, the counting term can not be directly expressed with an
arithmetic expression. This is because the value of the value of the
counting term is depends on the relationship between the values of
variables $x$ and $y$. In any model $\mathcal{M}$ where $x \le y$ is
true the formula $F$ is equivalent to $c = x$, and in any model $M$
where $x > y$ is true the formula $F$ is equivalent to $c = y$.
However, we can ``split'' on the relationship between $x$ and $y$ to
obtain a purely arithmetic formula equivalent to $F$
\begin{align*}
  (x \le y \implies c = x) \lor (x > y \implies c = y)\enspace.
\end{align*}
\end{example}

Counting constraints provide a way to specify how many integers
satisfy a given predicate and can, in a bounded setting, be used to
express quantifiers over the integers. An existential quantifier can
be transformed into a constraint involving a cardinality greater than
1, and a universal quantifier into a counting constraint composed of a
cardinality equal to 0, i.e.\footnote{Can we express $\forall k . 0 \le k \le 10$? The
proposed reduction turns this into $\cterm{k < 0 \lor k > 10} = 0$
in which the counting is not bounded and can evaluate to $+\infty$.}\todo{Check footnote!}
\begin{align*}
  \exists k . \phi &\;\;\equiv\;\; \cterm{\phi} > 0 \enspace, &
  \forall k . \phi &\;\;\equiv\;\; \cterm{\phi} = 0 \enspace.
\end{align*}

Reasoning with quantifiers in SMT is known to be a hard problem
\cite{ge2010solving,weispfenning1988complexity}. Some theories, such
as arithmetic, support quantifier elimination \cite{cooper}. But, the
complexity of quantifier elimination is prohibitive for practical
applications and may not be available for some theories of interest,
such as the theory of arrays \cite{bradley2006s}. Even when a theory
admits quantifier elimination,  simpler methods based on variable
instantiation may work better in practice \cite{dutertre2015solving}.

In our context, we are looking at a restricted fragment of formulas
built on the syntax of Figure \ref{formula}. In this fragment there is
no nested counting constraints or quantifier alternations, the main
sources of complexity in quantified reasoning. This allows us to
consider only atoms that are \emph{counting-univariate}, i.e. each
atom within the counting constraint contains a single occurrence of
the counting variable. Further, we only consider the case where the
only coefficients with the counting variable are $\pm 1$, i.e. atoms
all atoms are of the form $k < t$ or $k \geq t$, and atoms such as $2k <
x$ are not allowed. This is restriction is not fundamental as the
presented algorithms can be extended to cover them by relying on
divisibility constraints.

It can be shown that every formula in this class is equivalent to a
formula of the form
\begin{equation}
F(\vec{x}, \vec{c}) \land \bigwedge_i c_i = \cterm{\phi_i(k, \vec{x}, \vec{c})}\enspace,
\label{maingoal}
\end{equation}
where no counting constraints appear in $F$. In the following, we
assume all formulas are in above form.

\subsection{Decision Procedure}

We now present an algorithm that decides whether a formula of the form
\ref{maingoal} is satisfiable and if so produces a model.

\subsubsection{Overview}

We assume that we have an SMT solver \solver that can decide
satisfiability of formulas in theory $\qflia$, i.e. formulas without
counting constraints, and that it can produce models in the case when
the formula being checked is satisfiable. In addition, we assume that
solver \solver can be used in an incremental fashion, i.e. that new
assertions can be freely added and the it supports the \push and \pop
commands.\footnote{These are are standard featrues that all major SMT
solvers support.}

The decision procedure we describe will incrementally translate every
counting constraint to an equivalent arithmetic formula. In order to
do that, as in Example \ref{ex:disjunction}, the procedure will
introduce additional assumptions on the relationship between various
terms contained in the constraint. These assumptions, along with the
arithmetic formula that we incrementally construct, will, in the
limit, amount to a full translation of the counting constraint. But,
in practice, in order to find a model or show it's unsatisfiability,
one often needs to construct only a partial representation of the
counting constraint. This incremental model-drive construction is key
to practical effectiveness and will be guided by the models obtained
from the SMT solver, which are used to enumerate the sets of relevant
assumptions.

Our procedure works as follows: first, the formula without the
counting constraints is forwarded to the solver we already have. If it
is \texttt{unsat}, then the formula with the counting constraints is
also \texttt{unsat}. Otherwise, we get a model and we compute a
symbolic arithmetic expression for every counting constraints, under
the assumptions that hold in the model. Then, both the assumptions and
the value of the counting constraints are added as new clauses to the
solver. If $\mathcal{S}$ answers \texttt{sat}, then we have a model
which satisfies the formula with the counting constraints. Otherwise,
we restore the solver to its previous state, we assert the negation of
the assumptions and we retry the procedure.

A pseudo-code description is shown in algorithm \ref{arith}.

\begin{algorithm}[h]
\caption{Satisfiability of arithmetic formula with counting constraints}\label{arith}
\begin{algorithmic}[1]
\State \Call{assert}{$F(\mathbf{y}, c_1, \ldots, c_n)$}
\While{$\mathcal{M} = $ \Call{check-sat}{\ } }
    \State \Call{push}{\ }
    \State $A \gets \emptyset$
    \ForAll{ $i$ in $[1..n]$}
        \State $A_i, S_i \gets $ \Call{interpret-constraint}{$c_i$, $\phi_i$,
        $\mathcal{M}$}
        \State $A \gets A \cup A_i$
        \If{$S_i$ is infinite}
            \State \Call{assert}{$\lnot A$}
            \State \Call{continue}{}
        \EndIf
    \EndFor
    \State \Call{assert}{$A$}
    \State \textsc{assert}$\left(\bigwedge\limits_{i=1}^n c_i = \sum\limits_{[a, b] \in S_i} b - a\right)$
    \If {\Call{check-sat}{\ } }
        \State \Call{pop}{\ }
        \State \Return{sat}
    \EndIf
    \State \Call{pop}{\ }
    \State \Call{assert}{$\lnot A$}
\EndWhile
\State \Return{unsat}
\end{algorithmic}
\label{arith}
\end{algorithm}

We assume that our SMT solver $\mathcal{S}$ supports the following operations (besides the variable definitions):
\texttt{assert} (adds a formula to the current context),
\texttt{check-sat} (checks the satisfiability of the conjunction of the
formulas in the current context), \texttt{push} (creates a new context
with the current context formula), \texttt{pop} (restores the context to
the last \texttt{push} call). These operations are supported by most
modern SMT solvers which can work in an incremental way.


\begin{example}
\ \newline Let $F \equiv N > z \land N > u \land z \ge 10 \land c > z \land u \ge 0 \land c = \sharp\{ x \mid 0 \le x < N \land (x < z \lor x < u) \}$.
For this formula, the algorithm works as follows:
\begin{itemize}
\item \textsc{assert}$\left(N > z \land N > u \land z \ge 10 \land c > z \land u \ge 0 \right)$.
\item \textsc{push}$()$
\begin{itemize}
	\item First, get a model: $\mathcal{M} = \{ z \mapsto 10, N \mapsto 11, u \mapsto 0, c \mapsto 11\}$.
	\item Compute the domain of $\{ x \mid 0 \le x < N \land (x < z \lor x < u) \}$.
	\item The assumptions are $N > z, u < z, N > 0, N > u$ and the domain is $S = \{ [0, z) \}$.
	\item \textsc{assert}$\left(N > z \land u < z \land N > 0 \land N > u\right)$
	\item \textsc{assert}$\left(c = z)\right)$
	\item Then it is \texttt{unsat}, as we can't have both $c = z$ and $c> z$
\end{itemize}
\item \textsc{pop}()
\item The assumptions were not correct, we want to try something else.
\item \textsc{assert}$\left(\lnot \left(N > z \land u < z \land N > 0 \land N > u\right)\right)$
\item \textsc{push}$()$
\begin{itemize}
\item Get a new model: $\mathcal{M} = \{ z \mapsto 10, N \mapsto 30, u \mapsto 20, c \mapsto 11\}$.
\item Compute the domain and assumptions, it gets $A = \{ N \ge u, u > z, N > 0, u > 0\}$ and $S = [0, u)$.
\item \textsc{assert}$\left(N \ge u \land u > z \land N > 0 \land u > 0\right)$
\item \textsc{assert}$\left(c = u\right)$.
\item Now it is \texttt{sat}, and the model $\mathcal{M} = \{ z \mapsto 10, N \mapsto 30, u \mapsto 20, c \mapsto 20\}$ satisfies the whole formula.
\end{itemize}
\end{itemize}

\end{example}

\subsubsection{Constraints Interpretation}

In this section, we explain how a set $\sharp\{x\mid\phi(x)\}$ can be symbolically computed under
some assumptions.

\begin{definition}[Assumptions]

We call assumptions a set $A$ whose elements are literals of the theories
$\mathbf{T}_i$. In the context of a first order formula, writing $A$
means the conjunctions of the atoms of $A$.

\label{assumptions}

\end{definition}


\begin{definition}[Symbolic Interval]

A symbolic interval is a couple of values that are either arithmetic
variables, constants or $\infty$. If $I = [a, b)$, $x \in I$ is
defined as $x < b \land a \le x$. \newline~\newline
It is finite if none of the bounds are infinite.

\label{symbolic}

\end{definition}


\begin{definition}[Arithmetic Domain]

An arithmetic domain is a finite set of symbolic intervals (definition
\ref{symbolic}). It is  associated to some assumptions
(definition \ref{assumptions}) which make them disjoint and ensure that
the lower bound of an interval is lower than the upper bound.

\label{arithmetic}

\end{definition}

Let S be an arithmetic domain. We write $x \in S$ for
$\left(\bigvee\limits_{I \in S} x \in I\right)$.

Let $\psi_i$ be one of the formula of the counting constraints. We are going to give rules to compute $(S, A)$ such as
$A \implies \forall x \left(\psi_i(x, \mathbf{y}) \iff x \in S\right)$. The goal is to express the
cardinality $\sharp\{x \mid \psi_i(x)\}$ with an arithmetic formula (see below), which is easy if $S$ is only
comprised of distinct intervals. That is why we want the set $S$
to satisfy the following property:


\begin{property}[Distincts]

Let $S$ be an arithmetic domain computed with the rules described above, and $A$ the associated set of assumptions.
Let $I, J \in S$ such that $I \neq J$.

Then, $A \implies (x \in I \implies x \not\in J)$, i.e.
$\{x\ |\ x \in I\}$ and $\{x\ |\ x \in J\}$ are
disjoints\footnote{Here, $x \in I$ is still defined as $a \le x \land x < b$ if $I = [a, b)$.}.

\label{distincts}

\end{property}


The arithmetic domain and the associated set of assumptions is computed inductively on the formula
$\psi_i$.

\paragraph{Base cases}

If the formula is $y \le x$ where $y$ is an arithmetic variable, then the arithmetic domain is
$\{[y, +\infty)\}$ with no assumptions.

If the formula is $x < y$, then the arithmetic domain is $\{(-\infty, y)\}$, with no assumptions.

If the formula is a literal $l$ of a theory $\mathbf{T}_i$, then we look at the value of $l$ in
$\mathcal{M}$. If it is \texttt{true}, then $S = \{(-\infty, +\infty)\}$ and the assumptions are
$\{l\}$. If it is \texttt{false}, then $S = \emptyset$ and the assumptions are $\{\lnot l\}$.

Property \ref{distincts} clearly holds.

\paragraph{Conjunction}

Let $S$ and $S'$ be two arithmetic domains computed for $\phi$ and $\psi$, and $A$, $A'$ the sets of assumptions associated to
them. The domain of $\phi\land\psi$ is the intersection of $S$ and $S'$. To construct this intersection $S \sqcap S'$, we are going to
do the intersection of an interval $I$ of $S$ and an interval $J$
of $S'$.

So, let $I = [a, b)$ and $J = [c, d)$ be two symbolic intervals. We want a new
interval $K_{I, J} = [e, f)$ such as
$A \cup A' \cup A_{S \sqcap S'} \implies \left(x \in I \land x \in J \iff x \in K_{I, J}\right)$.
$A_{S \sqcap S'}$ is a set of additional assumptions that are added to make it possible to define
the
intersection. For instance, if the two intervals are $[0, x)$ and $[0, y)$, then $x < y$ or $y \le x$
must be added to claim that the intersection is either $[0, x)$ or $[0, y)$.

To choose what are the assumptions needed to compute the intersection, we are going to use the model
$\mathcal{M}$
of the formula $F$, so as the consistency between the assumptions is guaranteed. For any arithmetic variable $x$, $x_\mathcal{M}$ is
defined as the value of $x$ in this model, and more generally for any
term $t$ of theory $T_{\mathbb{Z}}$ (and $+\infty$, $-\infty$),
$t_{\mathcal{M}}$ is the interpretation of $t$ in this model.

%In what follows, we describe a way to compute $K_{I, j}$ as well as
%$A_{S \sqcap S'}$ in regards to the model, so as the assumptions are
%consistent (and the model satisfies them).

If
$max(a_\mathcal{M}, c_\mathcal{M}) < min(b_\mathcal{M}, d_\mathcal{M})$,
then $K_{I, J} = [max(a, c), min(b, d))$ (where $max(a, c)$ is $a$
if $a_\mathcal{M} > c_\mathcal{M}$, else $c$), else it is undefined
(because in the model, the interpretation of the intervals are indeed
disjoints).

The intersection of $I$ and $J$ is guided by the model.
Every time there is a decision to take (such as $a < c$), the values
of the model are looked at. Those decisions are recorded, as they
are the assumptions required for this interpretation to be correct. The
set $A_{S\sqcap S'}$ is composed of those decision. By induction, it is clear that if the same model
$\mathcal{M}$ is used for every intersection, then $\mathcal{M}$ satisfies $A \cup A' \cup A_{S \sqcap S'}$.

%It is also clear that by induction, if $S$ and $S'$ have the
%property \ref{distinct}, $S\sqcap S'$ also respect this property.

Then, the intersection of the domains is defined as
$S \sqcap S' = \{ K_{I, J} \ |\ I \in S, J \in S' \}$. Some intervals $K_{I, J}$ may not be defined
if they are empty. The associated assumptions set is $A \cup A' \cup A_{S \sqcap S'}$.

Property \ref{distincts} clearly holds if it holds for both $S$ and $S'$.

\begin{example}

Let $S = \{ I_1, I_2\} =  \{[0, 10), [12, z)\}$, $A = \{ z > 12 \}$, $S' = \{J\}= \{[15, y)\}$, $A =
\{ y > 15 \}$.

The model $\mathcal{M}$ is $\{ z \mapsto 16, y \mapsto 20 \}$.

The algorithm intersects $I_1$ with $J$,  which is $\emptyset$ as $12 > 10$. Then it intersects
$I_2$ with $J$. As $max(15, 12) = 15 < min(z_\mathcal{M}, y_\mathcal{M}) = 16 = z_\mathcal{M}$, the
intersection is $[15, z)$. The assumptions needed to do this intersection are $15 > 12, y > z, z >
15$.

The arithmetic domain is then $S\sqcap S' = \{[15, z)\}$ with the associated assumptions $\{z > 12,
y > 15, 15 > 12, y > z, z > 15\}$.

\end{example}

\paragraph{Negation}
We want to compute an arithmetic domain for a formula $\lnot \phi$. We assume we have an arithmetic
domain $S$ (with assumptions $A$) computed for $\phi$. Then property \ref{distincts} holds. There
are several cases, but here I only explain how the algorithm works if there is no $\infty$ in $S$.

By looking at the values in the model $\mathcal{M}$ of the bounds of the intervals in $S$, we can
choose inequalities that make $S = \{[a_1, b_1), \ldots, [a_k, b_k)\}$ ordered, i.e. $\forall i$, $A
\implies b_i < a_{i+1}$.

Then, the arithmetic domain for $\lnot \phi$ is $S' = \{(-\infty, a_1), [b_1, a_1), \ldots, [b_k,
+\infty)\}$, and the assumptions are those in $A$ and the new ones chosen to order $S$.

%\begin{definition}[Domains intersection, $S \sqcap S'$ and $A_{S \sqcap S'}$]

%If $S$ and $S'$ are two arithmetic domains associated to the
%assumptions $A$ and $A'$, then the intersection $S\sqcap S'$ is
%defined as an arithmetic domain such as\newline
%$A \cup A' \cup A_{S\sqcap S'} \implies \left(\left(x \in S \land x \in S'\right)\iff x \in S \sqcap S'\right)$\newline (where
%$x \in S$ means $\left(\bigwedge\limits_{I\in S} x \in I\right)$).

%\label{domains}

%\end{definition}

%\vspace{3mm}

%\begin{definition}[Complementary Domains, $S^c$]

%If $S$ is an arithmetic domain, then $S^c$ is defined as a set of
%intervals such as
%$\forall x.\ \left( \exists I \in S.\ x \in I\right) \iff \left( \not\exists I \in S^c.\ x \in I\right)$.

%\label{complementary}

%\end{definition}

%\vspace{3mm}


\begin{lemma}[Correctness]

Let $S$ be an arithmetic domain computed with the rules described above, and $A$ the associated set of assumptions.

Then,
$A \Rightarrow \left(\phi(x, \mathbf{y}) \iff x \in S\right)$.

\label{correctness-interpretation}

\end{lemma}

\begin{proof}
The lemma clearly holds for the base cases. It is correct by construction for the conjunction and the negation as there is an immediate correspondence with intersection and complementary domains.
Thus, the lemma holds by induction.
\end{proof}




A corollary of property \ref{distincts} is the fact that the cardinality of the reunion of the elements
of $S$ is the sum of the cardinality of its elements:
\begin{equation}
A \implies \sharp\{ x \in S \} = \sum\limits_{I \in S} \sharp\{x \in I\}
\end{equation}

Moreover, as the formula inside every counting constraints is of the form $t_1 \le x < t_2 \land
\phi$ (Figure \ref{formula}), it is clear that every interval in S is finite. Then, if $S$ and $A$
were computed for a formula $\phi$:
\begin{equation}
A \implies \sharp\{x \mid \phi(x, \mathbf{y})\} = \sum\limits_{[a, b)\, \in\, S} b - a
\end{equation}


\subsubsection{Termination and Correctness}

\begin{lemma}[Termination]

Algorithm \ref{arith} terminates.

\label{termination}

\end{lemma}

\begin{proof}
In the former section, we explained how the assumptions set was computed. An
assumption can be an equality, a disequality or an inequality between two terms
that appear in the formulas $\phi_i$. Thus, there is a finite number of possible
set of assumptions.

At every iteration, there is a $\text{\textsc{assert}}(\lnot A)$, hence the fact that there is
a finite number of iterations.
\end{proof}

\begin{lemma}[Correctness]

Algorithm \ref{arith} is correct.
\end{lemma}

\begin{proof}
Let $G = F \land \bigwedge\limits_{i} c_i = \sharp\{x\mid\phi(x)\}$ be a formula.

If algorithm \ref{arith} returns a model, then with lemma \ref{correctness-interpretation} and property
\ref{distincts} we can conclude that the model satisfies the formula, as every counting variable of
$F$ is set to the correct arithmetic expression.

If it returns \texttt{unsat}, and a model $\mathcal{M}$ satisfies $F$ nonetheless. Then $\mathcal{M}$
satisfies a set of assumptions which has been ruled out by the algorithm (else, as $\mathcal{M}
\vdash F$, it would be sent by the SMT solver), but the variables $c_i$ do not have a compatible
interpretation in $\mathcal{M}$ with the symbolic expression computed by the algorithm. Thus, it
contradicts lemma \ref{correctness-interpretation} or property \ref{distincts}.
\end{proof}

\subsection{Implementation}

I implemented the algorithm described above as an overlay on an existing SMT solver. Any solver
compatible with the SMT-LIB format \cite{barrett2010smt} and which can work in incremental mode can
be used.I mostly experimented with Yices and Z3. The code is available at
\url{https://github.com/xapantu/counting-smt/}.

The counting solver takes a file in the SMT-LIB format \cite{barrett2010smt}, with some extensions
to specify the constraints. It forwards most of the commands to the underlying solver. It only
extracts the syntax extension to specify counting constraints and execute the algorithm \ref{arith}
when it sees $\texttt{check-sat}$.

The counting solver and the SMT solver interacts via plain text SMT-LIB  commands, which is far from
optimal. The obvious benefits are that it is easier to implement and most SMT solvers are compatible
for free.
