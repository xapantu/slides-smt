% !TEX root = report.tex
% !TEX program = pdflatex

We will work in the setting of satisfiability modulo theories
\cite{barrett2009satisfiability}, specifically in the quantifier-free
theory of linear integer arithmetic $\qflia$. As usual, we assume a
set of variables ranging over $\mathbb{Z}$, and atoms as linear
constraints over these variables. A model $\mathcal{M}$ in $\qflia$ is
an assignment of variable to values in $\mathbb{Z}$. Given a model
$\mathcal{M}$, we can construct a new model where we assign a variable
$x$ to a value $v$ and the rest of the assignment unmodified, which we
denote with $\mathcal{M}^v_x$. As usual, when a formula $\phi$
evaluates to true in the model $\mathcal{M}$, we say that $\phi$ is
satisfied in $\mathcal{M}$ and denote this with $\mathcal{M} \vDash
\phi$. Given a term $t$ whose variables are defined in $\mathcal{M}$,
$t^\mathcal{M} \in \mathbb{Z}$ is the value of the term in the model.

We extend the usual language of $\qflia$ with counting constraints and
call this theory $\qfclia$, with formulas in this theory described in
Figure \ref{formula}.

\begin{figure}[h]
\begin{grammar}
<formulas> $\phi$ ::= $\phi$ $\land$ $\phi$
\alt $\phi$ $\lor$ $\phi$
\alt $\lnot$ $\phi$
\alt <atoms> of $\qflia$
\alt $c = \cterm{\phi}$
\end{grammar}

\caption{Formula syntax}
\label{formula}
\end{figure}

We call the terms of the form $\cterm{\phi}$ \emph{counting
terms}, and formulas of the form $c = \cterm{\phi}$
\emph{counting constraints}. 
Counting constraints denote the number of solutions for $k$
that satisfy the formula $\phi$.
More formally, given a model $\mathcal{M}$ that
assigns all variables other than $k$, the term $\cterm{\psi}$
denotes the cardinality of the set $D \subseteq \mathbb{Z}$ such that
$v \in D \iff \mathcal{M}^v_k \vDash \phi$. In order for the semantics
of the counting terms to be well defined, we assume that all counting
terms are over formulas that ensure $k$ is always bounded\footnote{Otherwise, as a counting term must be evaluated in $\mathbb{Z}$, no model can represent an infinite $D$. If a counting term has to be evaluated to the cardinality of an infinite set, then the formula is considered \texttt{unsat}.}.

\begin{example}
\label{ex:semantics}
The formula $c = \cterm{0 < k \le 10 \land 5 < k \le 10}$ is
satisfied in a model that assignes $c \mapsto 5$. It is

easy to see that in $\mathcal{M}$ the counting term evaluates to
\begin{align*}
  \cterm{0 < k \le 10 \land 5 < k \le 10}^\mathcal{M} = |D| = 5\enspace, &&
  \text{ where } D = [6, 10]\enspace.
\end{align*}

\end{example}

\begin{example}
\label{ex:simple}
The following formula
\begin{align*}
  c = \cterm{0 \le k \le c} \land (c > 0)
\end{align*}
is not satisfiable. In order for the formula to be satisfiable, the
the satisfying model $\mathcal{M}$ has to assign $c$ to a value
$c^\mathcal{M} > 0$. In such a model the counting term then evaluates to
\begin{align*}
  \cterm{0 \le k \le c}^\mathcal{M} = c^\mathcal{M} + 1\enspace,
\end{align*}
and the counting constraint is therefore must be false. Moreover it
is clear that the formula is equivalent to a purely arithmetic
formula $(c=c+1) \land (c>0)$, which is not satisfiable in $\qflia$.
\end{example}

Example above illustrates that, in the theory $\qfclia$, formulas with
counting constraints can be reduced to formulas in $\qflia$. This is
true due to $\qflia$ admitting quantifier elimination \cite{schweikardt}.
But, obtaining a full reduction is general is not as simple as above.

\begin{example}
\label{ex:disjunction}
Let  $F$ be the formula
\begin{align*}
  c = \cterm{0 < k \le x \land 0 < k \le y}\enspace.
\end{align*}
In this case, the counting term can not be directly expressed with a
single arithmetic term. This is because the value of the counting term
in a model $\mathcal{M}$ depends on the relationship between the values
of variables $x$ and $y$:
\begin{itemize}
  \item if $x^\mathcal{M} \le y^\mathcal{M}$, then $\cterm{0 < k \le x \land 0 < k \le y}^\mathcal{M} = x^\mathcal{M}$;
  \item if $x^\mathcal{M} > y^\mathcal{M}$, then $\cterm{0 < k \le x \land 0 < k \le y}^\mathcal{M} = y^\mathcal{M}$.
\end{itemize}
But, these assumptions on the relationship between $x$ and $y$ can
be used as symbolic assumption to obtain an equicalent arithmetic
formula
\begin{align*}
  (x \le y \implies c = x) \lor (x > y \implies c = y)\enspace.
\end{align*}
\end{example}

Counting constraints provide a way to specify how many integers
satisfy a given predicate and can be used to
express quantifiers over the integers. Existential and universal quantifiers can
be transformed into counting constraints. More specifically, a universally
quantified term $\forall k . \phi$ can be replaced by a fresh boolean variable
$b$, with the addition of those two additional constraints\footnote{A more simplistic reduction would be
  $\forall k . \phi \equiv \cterm{\lnot \phi} = 0$, but as there is no guarantee
  that $k$ is bounded, problems arise with formulas such as $\forall k . k > 0$,
  where $\cterm{k\leq 0}$ is then undefined (or if defined would need the
  comparison of an infinite number and $0$).}:
\begin{align*}
  \lnot b \implies \cterm{p < k < q \land \lnot \phi} > 0 \\
  b \implies \cterm{b \land \lnot \phi} = 0
\end{align*}
 where $p$ and $q$ are two fresh arithmetic variables. Rewriting existential
 quantification is done in a similar way.

Reasoning with quantifiers in SMT is known to be a hard problem
\cite{ge2010solving,weispfenning1988complexity}. Some theories, such
as arithmetic, support quantifier elimination \cite{cooper}. But, the
complexity of quantifier elimination is prohibitive for practical
applications and may not be available for some theories of interest,
such as the theory of arrays \cite{bradley2006s}. Even when a theory
admits quantifier elimination,  simpler methods based on variable
instantiation may work better in practice \cite{dutertre2015solving}.

In our context, we are looking at a restricted fragment of formulas
built on the syntax of Figure \ref{formula}. In this fragment there is
no nested counting constraints or quantifier alternations, the main
sources of complexity in quantified reasoning. This allows us to
consider only atoms that are \emph{counting-univariate}, i.e. each
atom within the counting constraint contains a single occurrence of
the counting variable. Further, we only consider the case where the
only coefficients with the counting variable are $\pm 1$, i.e. atoms
all atoms are of the form $k < t$ or $k \geq t$, and atoms such as $2k <
x$ are not allowed. This is restriction is not fundamental as the
presented algorithms can be extended to cover them by relying on
divisibility constraints.

It can be shown that every formula $F$ in this class is equivalent to a
formula of the form
\begin{equation}
\label{maingoal}
G(\vec{x}, \vec{c}) \land \bigwedge_i c_i = \cterm{\phi_i(k, \vec{x}, \vec{c})}\enspace,
\end{equation}
where no counting constraints appear in $A$, and all the formulas $\phi$
make sure that $k$ is always bounded. In the following, we assume all
formulas are in above form. When needed, we will denote with $F_\lia$
the arithmetic-only part of $F$, i.e. $F_\lia = G$.

\subsection{Decision Procedure}

We now present an algorithm that decides whether a formula of the form
\ref{maingoal} is satisfiable and if so produces a model.

\subsubsection{Overview}

We assume that we have an SMT solver \solver that can decide
satisfiability of formulas in theory $\qflia$, i.e. formulas without
counting constraints, and that it can produce models in the case when
the formula being checked is satisfiable. In addition, we assume that
solver \solver can be used in an incremental fashion, i.e. that new
assertions can be freely added (the \assert command) and that it
supports the \push and \pop commands.\footnote{These are all standard
features that all major SMT solvers support.}

The decision procedure we describe will incrementally translate every
counting constraint to an equivalent arithmetic formula. In order to
do that, as in Example \ref{ex:disjunction}, the procedure will
introduce additional assumptions on the relationship between various
terms contained in the constraint. These assumptions, along with the
arithmetic formula that we incrementally construct, will, in the
limit, amount to a full translation of the counting constraint. But,
in practice, in order to find a model, or show its unsatisfiability,
one often needs to construct only a partial representation of the
counting constraint. This incremental model-driven construction is key
to practical effectiveness and will be guided by the models obtained
from the SMT solver, which are used to enumerate the sets of relevant
assumptions.

Our procedure works as follows. First, the formula without the
counting constraints is asserted to the solver \solver. If it is
\unsat, then the formula with the counting constraints is also
\sat. Otherwise, we get a model $\mathcal{M}$, and we
compute a symbolic arithmetic expression for every counting term,
under the assumptions that hold in the model. Then, both the
assumptions and the value of the counting constraints are added as new
assertions to the solver. If \solver answers \texttt{sat}, then we
have a correct model which satisfies the formula, including the
counting constraints. Otherwise, we restore the solver to its previous
state, and assert the negation of the assumptions to eliminate the
model $\mathcal{M}$, and retry. Pseudo-code description of the
procedure is shown in Algorithm \ref{arith}.

\begin{algorithm}[h]
\caption{Satisfiability of arithmetic formula $F$ with counting constraints.}\label{arith}
\begin{algorithmic}[1]
\Procedure{\checkcsat}{F}
\Require $F \equiv G(\vec{x}, \vec{c}) \land \bigwedge_{i=1}^n c_i = \cterm{\phi_i(k, \vec{x}, \vec{c})}$.
\State $\assert(G)$
\While{$\mathcal{M} = \checksat()$}
    \LineComment Interpret constraints in $\mathcal{M}$ and collect assumptions.
    \State $\push()$
    \ForAll{ $i \in [1..n]$}
        \LineComment Compute assumptions and domains for $i$-th constraint
        \State $\langle A_i, D_i \rangle \gets \interpret(\mathcal{M}, c_i, \phi_i)$
        \State $\assert(A_i)$
        \State $\assert(c_i = \sum_{[a, b] \in D_i} b - a)$
        \If{$D_i$ is infinite}
			\State $\pop()$
            \State \Call{assert}{$\lnot A_i$}
            \State \Call{continue}{}
        \EndIf
    \EndFor
    \LineComment Check if the assumptions and counting constraints are consistent.
    \If {$\mathcal{M}_\sharp = \checksat()$}
        \State \Return{$\langle \sat, \mathcal{M}_\sharp \rangle$}
    \EndIf
    \State $\pop()$
    \State $\assert(\lnot A_1 \vee \ldots \vee \lnot A_n)$
\EndWhile
\State \Return{\unsat}
\EndProcedure
\end{algorithmic}
\label{arith}
\end{algorithm}

\newpage
\begin{example}

Consider the following formula

\begin{multline*}
\overbrace{N > z \land N > u \land z \ge 10 \land c > z \land u \ge 0}^G \\
\land c = \cterm{\underbrace{0 \le k < N \land (k < z \lor k < u)}_\phi}\enspace.
\end{multline*}

\begin{itemize}
\item First, we get a model of $G$. Let's assume it is $\mathcal{M} = \{ z \mapsto 10, N \mapsto 11, u \mapsto 0, c \mapsto 11\}$.
\item We want to adjust the model $\mathcal{M}$ to be consistent with the
counting constraints. To do that, we compute (symbolically) the value of the
counting term $\cterm{\underbrace{0 \le k < N \land (k < z \lor k < u)}_\phi}.$
\item In $\mathcal{M}$, this counting term evaluates to $z^\mathcal{M}$. Furthermore, under the assumptions $N > z, u < z, N > 0, N > u$, the counting term is equal to $z$.
\item Then, we ask for a new model with those assumptions, and the assertions $c = z$. The solver replies it is \texttt{unsat} (as we can not have $c > z$ and $c = z$).
\item Now, we retry, and we ask a model of $G \land \lnot (\ldots)$ (where $\ldots$ is the assumptions computed precedently).
\item We get a new model $\mathcal{M} = \{ z \mapsto 10, N \mapsto 30, u \mapsto
20, c \mapsto 11\}$. Evaluating the counting term while retaining the needed
assumptions gives $u$ with the assumptions $N \ge u, u > z, N > 0, u > 0$.
\item Again, we assert those assumptions, $c = u$ and ask for a model. This
time, we get a model $\mathcal{M} = \{ z \mapsto 10, N \mapsto 30, u \mapsto 20,
c \mapsto 20\}$, which satisfies the whole formula.
\end{itemize}

\end{example}

\subsubsection{Constraints Interpretation}

In this section, we explain how a set $\sharp\{x\mid\phi(x)\}$ can be
symbolically computed under some assumptions.

\begin{definition}[Assumptions]

We call assumptions a set $A$ whose elements are literals of the theory
$\qflia$. In the context of a first order formula, writing $A$
means the conjunctions of the atoms of $A$.

\label{assumptions}

\end{definition}


\begin{definition}[Symbolic Interval]

A symbolic interval is a couple of values that are either an arithmetic
term\footnote{In my implementation, the term has to be a variable or a constant,
which is sufficient if new variable can be added.} or $\infty$. If $I = [a, b)$, $x \in I$ is
defined as $x < b \land a \le x$. \newline~\newline
It is finite if none of the bounds are infinite.

\label{symbolic}

\end{definition}


\begin{definition}[Arithmetic Domain]

An arithmetic domain is a finite set of symbolic intervals (definition
\ref{symbolic}). It is  associated to some assumptions
(definition \ref{assumptions}) which make them disjoint and ensure that
the lower bound of an interval is lower than the upper bound.

\label{arithmetic}

\end{definition}

Let $D$ be an arithmetic domain. We write $x \lhd D$ for
$\left(\bigvee\limits_{I \in S} x \in I\right)$.

\begin{example}
Let $I_1 = [0, x), I_2 = [y, 10)$ be two symbolic intervals. Let $D = {I_1, I_2}$
an arithmetic domain, associated to the assumption $0 \leq x < y \leq 10$.

Then, $z \lhd D$ means that $z$ belongs either to $I_1$ or to $I_2$, that is:
\begin{align*} z \lhd D \iff 0 \leq z < x \lor y \leq z < 10\end{align*}
\end{example}

Let $\psi_i$ be one of the formula of the counting constraints. We are going to
give rules to compute a domain $D$ and a set of assumptions $A$ such as
$A \implies \forall x \left(\psi_i(x, \mathbf{y}) \iff x \lhd D\right)$. The goal is to express the
cardinality $\sharp\{x \mid \psi_i(x)\}$ with an arithmetic formula (see below), which is easy if $D$ is only
comprised of distinct intervals. That is why we want the set $D$
to satisfy the following property:


\begin{property}[Distincts]

Let $D$ be an arithmetic domain, and $A$ the associated set of assumptions.

$D$ is said to have the property \ref{distincts} if for $I, J \in D$ such that
$I \neq J$, we have $A \implies (x \in I \implies x \not\in J)$, i.e.
$\{x\ |\ x \in I\}$ and $\{x\ |\ x \in J\}$ are
disjoints\footnote{Here, $x \in I$ is still defined as $a \le x \land x < b$ if $I = [a, b)$.}.

\label{distincts}

\end{property}


The arithmetic domain and the associated set of assumptions is computed inductively on the formula
$\psi_i$.

\paragraph{Base cases}

If the formula is $y \le x$ where $y$ is an arithmetic term, then the arithmetic domain is
$\{[y, +\infty)\}$ with no assumptions.

If the formula is $x < y$, then the arithmetic domain is $\{(-\infty, y)\}$, with no assumptions.

If the formula is a literal $l$ of the theory $\qflia$, then we look at the value of $l$ in
$\mathcal{M}$. If it is \texttt{true}, then $S = \{(-\infty, +\infty)\}$ and the assumptions are
$\{l\}$. If it is \texttt{false}, then $D = \emptyset$ and the assumptions are $\{\lnot l\}$.

Property \ref{distincts} clearly holds.

\paragraph{Conjunction}

Let $D$ and $D'$ be two arithmetic domains computed for $\phi$ and $\psi$, and $A$, $A'$ the sets of assumptions associated to
them. The domain of $\phi\land\psi$ is the intersection of $D$ and $D'$. To construct this intersection $D \sqcap D'$, we are going to
do the intersection of an interval $I$ of $D$ and an interval $J$
of $D'$.

So, let $I = [a, b)$ and $J = [c, d)$ be two symbolic intervals. We want a new
interval $K_{I, J} = [e, f)$ such as
$A \cup A' \cup A_{D \sqcap D'} \implies \left(x \in I \land x \in J \iff x \in K_{I, J}\right)$.
$A_{S \sqcap S'}$ is a set of additional assumptions that are added to make it possible to define
the
intersection. For instance, if the two intervals are $[0, x)$ and $[0, y)$, then $x < y$ or $y \le x$
must be added to claim that the intersection is either $[0, x)$ or $[0, y)$.

To choose what are the assumptions needed to compute the intersection, we are going to use the model
$\mathcal{M}$
of the formula $G$, so as the consistency between the assumptions is guaranteed.

If
$max(a^\mathcal{M}, c^\mathcal{M}) < min(b^\mathcal{M}, d^\mathcal{M})$,
then $K_{I, J} = [max(a, c), min(b, d))$ (where $max(a, c)$ is $a$
if $a^\mathcal{M} > c^\mathcal{M}$, else $c$), else it is undefined
(because in the model, the interpretation of the intervals are indeed
disjoints).

The intersection of $I$ and $J$ is guided by the model.
Every time there is a decision to take (such as $a < c$), the values
of the model are looked at. Those decisions are recorded, as they
are the assumptions required for this interpretation to be correct. The
set $A_{D\sqcap D'}$ is composed of those decision. By induction, it is clear that if the same model
$\mathcal{M}$ is used for every intersection, then $\mathcal{M}$ satisfies $A \cup A' \cup A_{D \sqcap D'}$.

%It is also clear that by induction, if $S$ and $S'$ have the
%property \ref{distinct}, $S\sqcap S'$ also respect this property.

Then, the intersection of the domains is defined as
$D \sqcap D' = \{ K_{I, J} \ |\ I \in D, J \in D' \}$. Some intervals $K_{I, J}$ may not be defined
if they are empty. The associated assumptions set is $A \cup A' \cup A_{D \sqcap D'}$.

Property \ref{distincts} clearly holds if it holds for both $D$ and $D'$.

\begin{example}

Let $D = \{ I_1, I_2\} =  \{[0, 10), [12, z)\}$, $A = \{ z > 12 \}$, $D' = \{J\}= \{[15, y)\}$, $A =
\{ y > 15 \}$.

The model $\mathcal{M}$ is $\{ z \mapsto 16, y \mapsto 20 \}$.

The algorithm intersects $I_1$ with $J$,  which is $\emptyset$ as $12 > 10$. Then it intersects
$I_2$ with $J$. As $max(15, 12) = 15 < min(z_\mathcal{M}, y_\mathcal{M}) = 16 = z_\mathcal{M}$, the
intersection is $[15, z)$. The assumptions needed to do this intersection are $15 > 12, y > z, z >
15$.

The arithmetic domain is then $D\sqcap D' = \{[15, z)\}$ with the associated assumptions $\{z > 12,
y > 15, 15 > 12, y > z, z > 15\}$.

\end{example}

\paragraph{Negation}
We want to compute an arithmetic domain for a formula $\lnot \phi$. We assume we have an arithmetic
domain $D$ (with assumptions $A$) computed for $\phi$. Then property \ref{distincts} holds. There
are several cases, but here I only explain how the algorithm works if there is no $\infty$ in $D$.

By looking at the values in the model $\mathcal{M}$ of the bounds of the intervals in $D$, we can
choose inequalities that make $D = \{[a_1, b_1), \ldots, [a_k, b_k)\}$ ordered, i.e. $\forall i$, $A
\implies b_i < a_{i+1}$.

Then, the arithmetic domain for $\lnot \phi$ is $D' = \{(-\infty, a_1), [b_1, a_1), \ldots, [b_k,
+\infty)\}$, and the assumptions are those in $A$ and the new ones chosen to order $D$.


\begin{lemma}[Correctness]

Let $S$ be an arithmetic domain computed with the rules described above, and $A$ the associated set of assumptions.

Then,
$A \Rightarrow \left(\phi(x, \mathbf{y}) \iff x \lhd D\right)$.

\label{correctness-interpretation}

\end{lemma}

\begin{proof}
The lemma clearly holds for the base cases. It is correct by construction for the conjunction and the negation as there is an immediate correspondence with intersection and complementary domains.
Thus, the lemma holds by induction.
\end{proof}




A corollary of property \ref{distincts} is the fact that the cardinality of the reunion of the elements
of $D$ is the sum of the cardinality of its elements:
\begin{equation}
A \implies \sharp\{ x \lhd D \} = \sum\limits_{I \in D} \sharp\{x \in I\}
\end{equation}

Moreover, we can assume that every interval in $D$ is finite (otherwise this
special case is handled by the algorithm). Then, if $D$ and $A$
were computed for a formula $\phi$:
\begin{equation}
A \implies \sharp\{x \mid \phi(x, \mathbf{y})\} = \sum\limits_{[a, b)\, \in\, D} b - a
\end{equation}


\subsubsection{Termination and Correctness}

\begin{lemma}[Termination]

Algorithm \ref{arith} terminates.

\label{termination}

\end{lemma}

\begin{proof}
In the former section, we explained how the assumptions set was computed. An
assumption can be an equality, a disequality or an inequality between two terms
that appear in the formulas $\phi_i$. Thus, there is a finite number of possible
set of assumptions.

At every iteration, there is a $\text{\textsc{assert}}(\lnot A)$, hence the fact that there is
a finite number of iterations.
\end{proof}

\begin{lemma}[Correctness]

Algorithm \ref{arith} is correct.
\end{lemma}

\begin{proof}
Let $G = F \land \bigwedge\limits_{i} c_i = \sharp\{x\mid\phi(x)\}$ be a formula.

If algorithm \ref{arith} returns a model, then with lemma \ref{correctness-interpretation} and property
\ref{distincts} we can conclude that the model satisfies the formula, as every counting variable of
$F$ is set to the correct arithmetic expression.

If it returns \texttt{unsat}, and a model $\mathcal{M}$ satisfies $F$ nonetheless. Then $\mathcal{M}$
satisfies a set of assumptions which has been ruled out by the algorithm (else, as $\mathcal{M}
\vdash F$, it would be sent by the SMT solver), but the variables $c_i$ do not have a compatible
interpretation in $\mathcal{M}$ with the symbolic expression computed by the algorithm. Thus, it
contradicts lemma \ref{correctness-interpretation} or property \ref{distincts}.
\end{proof}

\subsection{Implementation}

I implemented the algorithm described above as an overlay on an existing SMT solver. Any solver
compatible with the SMT-LIB format \cite{barrett2010smt} and which can work in incremental mode can
be used.I mostly experimented with Yices and Z3. The code is available at
\url{https://github.com/xapantu/counting-smt/}.

The counting solver takes a file in the SMT-LIB format \cite{barrett2010smt}, with some extensions
to specify the constraints. It forwards most of the commands to the underlying solver. It only
extracts the syntax extension to specify counting constraints and execute the algorithm \ref{arith}
when it sees $\texttt{check-sat}$.

The counting solver and the SMT solver interacts via plain text SMT-LIB  commands, which is far from
optimal. The obvious benefits are that it is easier to implement and most SMT solvers are compatible
for free.
