% !TEX root = report.tex
% !TEX program = pdflatex

This report describes work I did during an internship at SRI
International, in Menlo Park, CA. During this internship, I worked on
decision procedures related to symbolic model checking of parametric
systems that are infinite-state and fault-tolerant, under the
supervision of Dejan Jovanović and Bruno Dutertre.

Model checking is a set of techniques that aim to establish whether a
system, described as a state-transition system, satisfies a property.
Both the system and the property are usually expressed in a suitable
logical formalism, such as first-order logic, while the properties
sometimes also includes temporal aspects such as ``always'' or
``eventually''. A fundamental class of properties is the case of
\emph{safety} properties, where the property $P$ expresses an
invariant of a system $S$. That is, we are interested in checking
whether $P$ is true in all the reachable states of the system $S$. For
infinite-state systems, the most common ways of proving/dis-proving a
property are based on techniques such as bounded model checking
\cite{biere2003bounded}, $k$-induction \cite{sheeran2000checking}, or
IC3/PDR \cite{pdr}. In the case of infinite-state systems, these
techniques start with a symbolic representation of the property $P$
and of the system $S$ in first-order logic and rely on satisfiability
modulo theories (SMT) solvers \cite{barrett2009satisfiability} as
reasoning engines.

SRI's computer science laboratory is developing a model checker,
called Sally\footnote{\url{http://sri-csl.github.io/sally/}}, that
implements bounded model checking, $k$-induction, and a variant of IC3
\cite{jovanovic2016property}. One of the main applications of Sally is
verification of fault-tolerant protocols. Currently, Sally can handle
protocols with a fixed number of component, but it is being extended
to parametric protocols, where the number of components is not fixed a
priori. In this context, a key issue is to be able to model and reason
symbolically about constraints that can express ``counting'' the
number of solutions. The ability to reason about number of solutions
is specially important in modeling of fault-tolerant protocols. A
typical example of counting constraints is an assumption of the form
``less than one third of the processes are faulty".

\todo{Add example of a system and a property.}

Several methods and tools have already been proposed to study
parametric protocols. However, they are not particularly suited to
check fault-tolerant protocols. For instance, \textsc{Cubicle}
\cite{ConchonGKMZ12} is suited to study a class of parameterized
protocols including cache coherence protocols, but is not equipped to
deal with counting constraints. On the other hand, recent developments
of the MCMT framework \cite{AlbertiGP16} allow for counting
constraints. But, the proposed decision procedures are mostly
theoretical and can not be integrated well into an invariant-discovery
algorithm like IC3 that requires advanced features such as
interpolation and generalization.

During my internship, I first implemented a decision procedure for
counting constraints in arithmetic, and then extended it to counting
constraints that can also support arrays. This decision procedure can
then be used with Sally to check parametric fault-tolerant protocols.
Both decision procedures are implemented as overlays to an existing
SMT solver making them readily applicable on top of existing
tecnology. The decision procedure for arithmetic is described in
Section \ref{sec:arith}. Section \ref{sec:arrays} presents the
extension of this decision procedure to arrays.

\ifthenelse{0=0}{}{

Model checking is used to prove properties of a system. A system is modeled by a set
of initial states and transitions. State transitions, initial states and properties are expressed in
first order logic. We are interested in infinite state systems. As it is not possible to enumerate
every state and verify they satisfy the properties, the goal is to find invariants of the system,
which are strengthening of the properties. Several techniques can be used to solve this problem,
such as bounded model checking, IC3 \cite{pdr}, or k-induction. For infinite state systems, model
checking is done symbolically and uses satisfiability modulo theories (SMT) solvers.

Extending this to support analysis of parametric systems is of major interest to prove correctness
of a large number of protocols, such as fault tolerant protocols\cite{AlbertiGP16} or cache coherence
protocols\cite{ConchonGKMZ12}. A key issue is handling efficiently cardinality constraints in
SMT.

During my internship, I worked on a decision procedure for counting constraints in arithmetic
(section 1), and then on an extension for counting constraints with arrays. Both decision
procedures were implemented as an overlay to an existing SMT solver.

I did my internship at SRI, California, under the supervision of Bruno Dutertre and Dejan
Jovanović.}
