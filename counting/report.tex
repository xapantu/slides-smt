\documentclass[]{article}
\usepackage{lmodern}
\usepackage{amssymb,amsmath}
\usepackage{ifxetex,ifluatex}
\usepackage{fixltx2e} % provides \textsubscript
% use upquote if available, for straight quotes in verbatim environments
\IfFileExists{upquote.sty}{\usepackage{upquote}}{}
% use microtype if available
\IfFileExists{microtype.sty}{%
\usepackage{microtype}
\UseMicrotypeSet[protrusion]{basicmath} % disable protrusion for tt fonts
}{}
\usepackage{hyperref}
\hypersetup{unicode=true,
            pdftitle={Solving Counting Constraints on First Order Formulas in SMT},
            pdfborder={0 0 0},
            breaklinks=true}
\urlstyle{same}  % don't use monospace font for urls
\IfFileExists{parskip.sty}{%
\usepackage{parskip}
}{% else
\setlength{\parindent}{0pt}
\setlength{\parskip}{6pt plus 2pt minus 1pt}
}
\setlength{\emergencystretch}{3em}  % prevent overfull lines
\providecommand{\tightlist}{%
  \setlength{\itemsep}{0pt}\setlength{\parskip}{0pt}}
\setcounter{secnumdepth}{5}
\usepackage{tikz}
\usepackage{bussproofs}
\usepackage{syntax}
\usepackage{amsthm}
\usepackage{color}
\usepackage{algorithm}
\usepackage[noend]{algpseudocode}
\setcounter{secnumdepth}{3}

\title{Solving Counting Constraints on First Order Formulas in SMT}
\date{August 28, 2016}

\begin{document}
\maketitle

{
\setcounter{tocdepth}{3}
\tableofcontents
}

\newtheorem{definition}{Definition} \newtheorem{lemma}{Lemma}
\newtheorem{theorem}{Theorem} \newtheorem{property}{Property}
\newtheorem{example}{Example}

\section*{Introduction}\label{introduction}
\addcontentsline{toc}{section}{Introduction}

%TODO

%\section{Model Checking of parametrized and fault tolerant systems}

%TODO

%\subsection{Counting Constraints}

%Model checking usually relies on checking that the negation of a
%property is unsatisfiable. Properties and transitions can be expressed
%with first order formulas, with or without quantifiers. This problem is
%known to be undecidable \cite{apt1986limits}, that
%is why it is crucial to understand what is the class of formulas used to
%specify them, and to restrict them if possible.

%Counting constraints provide a way to specify how many integers satisfy
%a given predicate. Thus, it is a generalization of quantifiers over the
%integers. For instance, if $a$ is an array of booleans of size $n$,
%the counting constraint $\sharp\{ x \ |\ a[x]\} > n/2$ specify that
%more than half the array is set to \texttt{true}. An existential
%quantifier can be transformed into a constraint involving a cardinality
%greater than 1, and a universal quantifier into a counting constraint
%composed of a cardinality equal to 0 (in this case, the predicate will
%be the negation of the formula).
%
%For the class of problems we want to study, they are very adapted to
%express properties of the systems, such as the maximum fraction of
%faulty components. However, dealing with several nested quantifiers is
%hard, and a lot of problems can be modeled with a single level of
%quantification (i.e.~there is no nested counting constraints terms
%inside other counting constraints). Thus, this problem can be restricted
%to this case and still be interesting, and provides a simplified version
%of quantifiers, while avoiding the requirement to support a more
%complete class of formulas. SMT formulas with a lot of quantifiers are
%indeed known to be hard to solve \cite{ge2009complete}.

\section{Counting Constraints for Arithmetic}

We consider a set of distinct theories $\mathbf{T}_\mathbb{Z}$,
$\mathbf{T}_{eq}$, $\mathbf{T}_2$, \ldots{}, $\mathbf{T}_n$, where
$\mathbf{T}_\mathbb{Z}$ is the theory of $\mathbb{Z}$, and
$\mathbf{T}_{eq}$ is the theory of equality between the other theory.
Atoms and variables of these theories are defined as usual. In the
following, a first order formula without quantifiers is defined with the
grammar \ref{formula}. The terms of the form
$c = \sharp\{ x\ |\ \psi(x) \}$ are called counting constraints. 
$\sharp\{ x\ |\ \psi(x)\}$ denotes the cardinality of a subset $S$
of $\mathbb{Z}$ such as $x \in S \iff \psi(x)$.

\begin{figure}[h]
\begin{grammar}
    
<formulas> $\phi$ ::= $\phi$ $\land$ $\phi$
\alt $\phi$ $\lor$ $\phi$
\alt $\lnot$ $\phi$
\alt <atoms> of $\mathbf{T}_i$
\alt $c = \sharp\{ x \mid y\ \text{\textless}\ x \le z \land \phi(x) \}$ where $c, y, z$ are variables of $\mathbf{T}_\mathbb{Z}$

\end{grammar}

\caption{Formula syntax}
\label{formula}
\end{figure}

\begin{example}
The formula $c = \{x \mid 0 < x \le 10 \land 5 < x \le 10\}$ is equivalent to the formula $c = 5$.
\end{example}

\subparagraph{Example 2} The formula $c = \{x \mid 0 < x \le c + 1\} \land c > 0$ is equivalent to the formula $c = c+1 \land c >0$ and is not satisfiable.

\subparagraph{Example 3} Let  $F \equiv c = \{x \mid 0 < x \le y \land 0 < x \le z\}$. In this case, $c$ can not be directly expressed with an arithmetic expression, as the symbolic value depends on wether $y < z$. However, with a disjonction on $y >z$, $F$ is also equivalent to a pure arithmetic formula $(y > z \implies c = y) \lor (y \le z \implies c = z)$.

Counting constraints provide a way to specify how many integers satisfy
a given predicate. Thus, it is a generalization of quantifiers over the
integers. An existential
quantifier can be transformed into a constraint involving a cardinality
greater than 1, and a universal quantifier into a counting constraint
composed of a cardinality equal to 0 (in this case, the predicate will
be the negation of the formula).

However, dealing with quantifiers in SMT formula is known to be a hard problem. Some theories
support quantifiers elimination, such as arithmetic \cite{cooper}, but these algorithms may not be very
efficient (even for arithmetic it can be better to instanciate the quantifiers \cite{dutertre2015solving}) and quantifiers elimination is not possible in every theory \cite{bradley2006s}.

That's why we look at a restricted subset of formulas built on the syntax of Figure \ref{formula}:
there is no nested counting constraints. Furthermore, to keep the algorithm simple, if the
quantified variable is $x$, every atom of the counting constraint where $x$ appears must be of the form $x < t$ or $t \le x$,
where $t$ is an arithmetic term where $x$ does not appear. Without loss of generality, we only
consider the case where $t$  is an arithmetic variable (a new variable equal to an arbitrary term
can always be added to the formula).

Moreover, it can be shown that every formula in this class is equivalent to a formula of the form:
\begin{equation}
F(c_1, \ldots, c_n, \mathbf{y}) \land \bigwedge\limits_{i=1}^n c_i = \sharp\{ x \ |\ \psi_i(x, c_1, \ldots, c_n,
\mathbf{y})\}
\label{maingoal}
\end{equation}
where no counting constraints appear in $F$.

We are going to explain a way to decide this formula, i.e. saying wether
they are satisfiable, and if yes give a model that satisfies this
formula.

\subsection{Decision Procedure}
\subsubsection{Overview}

We assume that we already have a solver to solve formulas without
counting constraints (which can give a model if the formula is
\texttt{sat}).

The algorithm translates every counting constraints in arithmetic expressions. To do that, it needs
to make additional assumptions (example 3). We use models constructed by the SMT solver to enumerate every possible set of
assumptions compatible with formula $F$.

It works as follows: first, the formula without the counting constraints is forwarded to
the solver we already have. If it is `unsat`, then the formula with the counting constraints is also
`unsat`. Else, we get a model. Then, the algorithm computes a symbolic arithmetic expression for every counting
constraints, under the assumptions that hold in the model. Then, both the assumptions and the value
of the counting constraints are added as new clauses to the solver. If it answers `sat`, then we
have a model which satisfies the formula with the counting constraints. Else, the solver is
restored to it previous state, the negation of the assumptions are added to the solver and we
retry the procedure.

A pseudo-code description can be seen in algorithm \ref{arith}.

\begin{algorithm}[h]
\caption{Satisfiability of arithmetic formula with counting constraints}\label{arith}
\begin{algorithmic}[1]
%\Procedure{MyProcedure}{}
\State \Call{assert}{$F(\mathbf{y}, c_1, \ldots, c_n)$}
\While{$\mathcal{M} = $ \Call{check-sat}{\ } }
    \State \Call{push}{\ }
    \State $A \gets \emptyset$
    \ForAll{ $i$ in $[1..n]$}
        \State $A_i, S_i \gets $ \Call{interpret-constraint}{$c_i$, $\phi_i$,
        $\mathcal{M}$}
        \State $A \gets A \cup A_i$
        \If{$S_i$ is infinite}
            \State \Call{assert}{$\lnot A$}
            \State \Call{continue}{}
        \EndIf
    \EndFor
    \State \Call{assert}{$A$}
    \State \textsc{assert}$\left(\bigwedge\limits_{i=1}^n c_i = \sum\limits_{[a, b] \in S_i} b - a\right)$
    \If {\Call{check-sat}{\ } }
        \State \Call{pop}{\ }
        \State \Return{sat}
    \EndIf
    \State \Call{pop}{\ }
    \State \Call{assert}{$\lnot A$}
\EndWhile
\State \Return{unsat}
\end{algorithmic}
\label{arith}
\end{algorithm}

\begin{example}
\end{example}
\subsubsection{Constraints
Interpretation}\label{constraints-interpretation}

\begin{definition}[Assumptions]

We call assumptions a set $A$ whose elements are litterals of the theories
$\mathbf{T}_i$. In the context of a first order formula, writing $A$
means the conjuctions of the atoms of $A$.

\label{assumptions}

\end{definition}

\vspace{3mm}

\begin{definition}[Symbolic Interval]

A symbolic interval is a couple of values that are either arithmetic
variables, constants or $\infty$. If $I = [a, b)$, $x \in I$ is
defined as $x < b \land a \le x$. \newline~\newline
It is said to be finite if none of the bounds are infinite.

\label{symbolic}

\end{definition}

\vspace{3mm}

\begin{definition}[Arithmetic Domain]

An arithmetic domain is a finite set of symbolic intervals (definition
\ref{symbolic}). It is  associated to some assumptions
(definition \ref{assumptions}) which make them disjoint and ensure that
the lower bound of an interval is lower than the upper bound.

\label{arithmetic}

\end{definition}

Let S be an arithmetic domain. We write $x \in S$ for
$\left(\bigvee\limits_{I \in S} x \in I\right)$.

Let $\psi_i$ be one of the formula of the counting constraints. We are going to give rules to compute $(S, A)$ such as
$A \implies \forall x \left(\psi_i(x, \mathbf{y}) \iff x \in S\right)$.

\paragraph{Base cases}
\paragraph{Negation}

\paragraph{Intersection}

Let $S$ and $S'$ be two arithmetic domains, and $A$, $A'$ the sets of assumptions associated to them. Thus, to do the intersection of the domains $S \sqcap S'$, we are going to
do the intersection of an interval $I$ of $S$ and an interval $J$
of $S'$.

So, let $I = [a, b)$ and $J = [c, d)$ be two symbolic intervals. We want a new
interval $K_{I, J} = [e, f)$ such as
$A \cup A' \cup A_{S \sqcap S'} \implies \left(x \in I \land x \in J \iff x \in K_{I, J}\right)$.
$A_{S \sqcap S'}$ is a set of additional assumptions that are added to make it possible to define an
intersection. For instance, if the two intervals are $[0, x)$ and $[0, y)$, $x < y$ or $y \le x$
must be added to claim that the intersection is either $[0, x)$ or $[0, y)$.

To decide what are the assumptions needed to compute the intersection, let's assume we have a model
$\mathcal{M}$ of the formula $F$. For any arithmetic variable $x$, $x_\mathcal{M}$ is
defined as the value of $x$ in this model, and more generally for any
term $t$ of theory $T_{\mathbb{Z}}$ (and $+\infty$, $-\infty$),
$t_{\mathcal{M}}$ is the interpretation of $t$ in this model.

In what follows, we describe a way to compte $K_{I, j}$ as well as
$A_{S \sqcap S'}$ in regards to the model, so as the assumptions are
consistents (and the model satifies them).

Let $I = [a, b) \in S$ and $J = [c, d) \in S'$, if
$max(a_\mathcal{M}, c_\mathcal{M}) < min(b_\mathcal{M}, d_\mathcal{M})$,
then $K_{I, J} = [max(a, c), min(b, d))$ (where $max(a, c)$ is $a$
if $a_\mathcal{M} > c_\mathcal{M}$, else $c$), else it is undefined
(because in the model, the interpretation of the intervals are indeed
disjoints).

So, this is the intersection of $I$ and $J$ guided by the model.
Every time there is a decision to take (such as $a < c$), the values
of the model are looked at. Those decisions must be recorded, as they
are the assumptions required for this interpretation to be correct. The
set $A_{S\sqcap S'}$ is composed of those decision. It is clear that
the model $\mathcal{M}$ satisfies the set $A_{S \sqcap S'}$ (as well
as $A$ and $A'$ by induction).

%It is also clear that by induction, if $S$ and $S'$ have the
%property \ref{distincts}, $S\sqcap S'$ also respect this property.

Then, the intersection of the domains can be defined as
$S \sqcap S' = \{ K_{I, J} \ |\ I \in S, J \in S' \}$. Some intervals $K_{I, J}$ may not be defined
if they are empty.

\begin{lemma}[Distincts]

Let $S$ be an arithmetic domain computed with the rules described above, and $A$ the associated set of assumptions.
Let $I, J \in S$ such that $I \neq J$.

Then, $A \implies (x \in I \implies x \not\in J)$. Thus,
$\{x\ |\ x \in I\}$ and $\{x\ |\ x \in J\}$ are
disjoints\footnote{Here, $x \in I$ is still defined as $a \le x \land x < b$ if $I = [a, b)$.}.

\label{distincts}

\end{lemma}

\vspace{4mm}

%\begin{definition}[Domains intersection, $S \sqcap S'$ and $A_{S \sqcap S'}$]

%If $S$ and $S'$ are two arithmetic domains associated to the
%assumptions $A$ and $A'$, then the intersection $S\sqcap S'$ is
%defined as an arithmetic domain such as\newline
%$A \cup A' \cup A_{S\sqcap S'} \implies \left(\left(x \in S \land x \in S'\right)\iff x \in S \sqcap S'\right)$\newline (where
%$x \in S$ means $\left(\bigwedge\limits_{I\in S} x \in I\right)$).

%\label{domains}

%\end{definition}

%\vspace{3mm}

%\begin{definition}[Complementary Domains, $S^c$]

%If $S$ is an arithmetic domain, then $S^c$ is defined as a set of
%intervals such as
%$\forall x.\ \left( \exists I \in S.\ x \in I\right) \iff \left( \not\exists I \in S^c.\ x \in I\right)$.

%\label{complementary}

%\end{definition}

%\vspace{3mm}


\begin{lemma}[Correctness]

Let $S$ be an arithmetic domain computed with the rules described above, and $A$ the associated set of assumptions.

Then,
$A \Rightarrow \left(\phi(x, \mathbf{y}) \iff x \in S\right)$.

\label{}

\end{lemma}
\subsubsection{Termination and Correctness}

\begin{lemma}[Termination]

Algorithm \ref{arith} terminates.

\label{termination}

\end{lemma}

\begin{proof}
In the former section, we explained how the assumptions set was computed. An
assumption can be an equality, a disequality or an inequality between two terms
that appear in the formulas $\phi_i$. Thus, there is a finite number of possible
set of assumptions.

At every iteration, there is a $\text{\textsc{assert}}(\lnot A)$, hence the fact that there is
a finite number of iterations.
\end{proof}

\begin{proof}
\end{proof}

%\begin{lemma}[Consistency]

%Let $A$ is a set of assumptions computed with the rules described above. Then it is consistent, i.e. the
%conjunction of it elements is satisfiable.

%\label{consistency}

%\end{lemma}

%\vspace{4mm}


%\subsubsection{Interpret a constraint with a
%model}\label{interpret-a-constraint-with-a-model}

%We are now interested in how we can write an algorithm which provides an
%arithmetic domain and a set of assumptions from a formula, following the
%inference rules of the former section. Hence, the only two missing
%operations are building the domains $S^c$ and $S \sqcap S'$. This
%algorithm must be correct, but we also want it to ensure several
%properties, so as the resulting domains can then be used to compute the
%cardinality at no further cost. These properties are on both the domain
%and the assumptions associated to it.

%This algorithm uses a model (i.e.~an assignment of the free variables of
%the formula $F$ of equation \ref{maingoal} whose it is a model of),
%and computes the arithmetic domain in respect to this model.
%
%\vspace{3mm}

%If the property \ref{distincts} holds and every interval of $S$ is
%finite, then:

%$A \implies \sharp\{ x \in S \} = \sum\limits_{[a, b) \in S} b - a$.


%\subsubsection{Algorithm to solve arithmetic counting
%contraints}\label{algorithm-to-solve-arithmetic-counting-contraints}

%We describe an algorithm to solve a formula:

%\begin{equation}
%F(\mathbf{y}, c_1, \ldots, c_n) \land
%\bigwedge_{i = 1} ^n c_i = \sharp\{x\ |\ \phi_i(\mathbf{y}, c_1, \ldots, c_n, x)\}
%\end{equation}

%where $F$ does not contain counting constraints.

%We assume we have an SMT solver that can solve formulas written with the
%theories
%$\mathbf{T}_\mathbb{Z}, \mathbf{T}_{eq}, \mathbf{T}_1, \ldots, \mathbf{T}_m$.
%It needs to support some operations (besides the variable definitions):
%\texttt{assert} (adds a formula to the current context),
%\texttt{check-sat} (checks the satisfiability of the conjunction of the
%formulas in the current context), \texttt{push} (creates a new context
%with the current context formula), \texttt{pop} (restores the context to
%the last \texttt{push} call). These operations are supported by most
%modern SMT solvers which can work in an incremental way.
%
%The algorithm \ref{arith} works as follows: it asks for a model
%$\mathcal{M}$ of the formula $F$, then interpret every counting
%constraints to a symbolic expression under some assumptions. The
%equality between those expressions and the $c_i$, as well as the
%assumptions are then enforced with an \texttt{assert}. Then, if the
%solver says it is satisfiable, the values of the $c_i$ in the new
%model respect the counting constraints equations. If it is not, then it
%means that the assumptions $A$ and the counting constraints equations
%are not consistent, thus $\lnot A$ can be asserted, and we can re-try.
%
%\vspace{3mm}

%\begin{lemma}[Termination]
%
%Algorithm \ref{arith} terminates.
%
%\label{termination}
%
%\end{lemma}
%
%\begin{proof}
%In the former section, we explained how the assumptions set was computed. An
%assumption can be an equality, a disequality or an inequality between two terms
%that appear in the formulas $\phi_i$. Thus, there is a finite number of possible
%set of assumptions.
%
%At every iteration, there is a $\text{\textsc{assert}}(\lnot A)$, hence the fact that there is
%a finite number of iterations.
%\end{proof}
%
%\vspace{3mm}
%
%\begin{lemma}[Correctness]
%
%\label{correctness}
%
%\end{lemma}
%
%\begin{proof}TODO\end{proof}
%
%\paragraph{Example}\label{example}
%
%TODO
%
\subsection{Implementation}

\section{Solving Counting Constraints with
Arrays}\label{solving-counting-constraints-with-arrays}

In this section, we describe an extension of the previous algorithm to
deal with arrays. The syntax for $k$ arrays $a_1, \ldots, a_k$ is
described in Figure \ref{syntaxarray}. It is important to note that
arrays are only accessed on the quantified variable, and not on a
general term built on this variable (such as $x + 1$, or a nested
array rerad). Removing this syntax restriction leads to an undecidable
array theory fragment, as stated in
\cite{bradley2006s}, even for small additions to the
fragment.

An array has a size, which is an arithmetic variable of the theory
$\mathbf{T}_\mathbb{Z}$. This is similar to
\cite{AlbertiGP16} (with the subtle difference that
different arrays can have different size) but unlike
\cite{ConchonGKMZ12}, whose fragment does not
provide a syntax to express array length. In the context of fault
tolerant systems, this is an important detail, as we typically want to
specify that a fraction of the systems can be faulty.

As soon as there is an array term in the counting constraints, the
cardinality can no longer be infinite, as the array can only be accessed
on a finite interval.

It may be interesting to have array reads outside of the counting
constraints, but they can be rewritten as counting constraints, like in
\cite{bradley2006s} or
\cite{AlbertiGP16}. In the next section we explain
how this algorithm can be changed to work with the usual arrays of an
SMT solver.

\begin{figure}[h]
\begin{grammar}
    
<counting constraints> $\psi(x)$ ::= $\psi(x)$ $\land$ $\psi(x)$
\alt $\psi(x)$ $\lor$ $\psi(x)$
\alt $\lnot$ $\psi(x)$
\alt <atoms> of $\mathbf{T}_i$ where $x$ does not appear
\alt $x \leq y$ where $y$ is a variable of $\mathbf{T}_\mathbb{Z}$
\alt $y \textless x$ where $y$ is a variable of $\mathbf{T}_\mathbb{Z}$
\alt $\phi(a_1[x], \ldots, a_k[x])$ where $\phi$ does not have counting constraints


\end{grammar}
\caption{Array Extension}
\label{syntaxarray}
\end{figure}

\subsection{Algorithm}\label{algorithm}

The algorithm to solve counting constraints with arrays is mostly the
same as algorithm \ref{arith}, the main difference is that the
constraints on the arrays must be saved and then be consistent.

During my internship, I experimented several possible algorithms to
manipulate those constraints. The algorithm I describe here migth seem a
bit brutal as it extensively rely on the underlying SMT solver, but it
worked better than the other attempts, probably because a modern SMT
solver can be much more efficient than a less optimized specialized
algorithm.

\subsubsection{Arithmetic And Arrays
Domains}\label{arithmetic-and-arrays-domains}

\begin{definition}[Array Constraint]

An array constraint is a first order, quantifier free, formula whose
free variables are the free variables of the formula and the variables
$a_1[\cdot], \ldots, a_k[\cdot]$.

\label{array}

\end{definition}

\vspace{3mm}

\begin{definition}[Domain]

An domain is a finite set of symbolic intervals (definition
\ref{symbolic}), every one of them associated to an array constraint
(definition \ref{array}).

\label{domain}

\end{definition}

An arithmetic domain (definition \ref{arithmetic}) can then be seen as a
domain whose every symbolic interval is associated to the constraint
\texttt{true}.

Figure \ref{arraybases} introduces the rules for the base cases of
domain computations. The conjonction and the negation are still done
according to the rules \ref{not} and \ref{and}.

\begin{figure}[h]
\begin{prooftree}
\AxiomC{}
\UnaryInfC{$((-\infty, +\infty), \phi(a_1[x], \ldots, a_k[x])), \emptyset \vdash \phi(a_1[x], \ldots, a_k[x])$}
\end{prooftree}
\begin{prooftree}
\AxiomC{}
\UnaryInfC{$([y; +\infty), \top), \emptyset \vdash y \leq x$}
\end{prooftree}

\begin{prooftree}
\AxiomC{}
\UnaryInfC{$((-\infty; y), \top), \emptyset \vdash x < y$}
\end{prooftree}

\caption{Base Cases}
\label{arraybases}
\end{figure}

So, we need to redefine the operations of intersection ($S \sqcap S'$)
and negation ($S^c$) for those domains.

\subsubsection{Intersection and negation of the
domains}\label{intersection-and-negation-of-the-domains}

Let $S$ and $S'$ two domains. The intersection is done the same way
it is done for arithmetic domains, but when intersecting two symbolic
intervals, if the resulting intervals is not empty, then we associate to
the result the conjonction of the two array constraints.

\subsubsection{Enforcing the Domains}\label{enforcing-the-domains}

We now want to generate a set of constraints that are equivalent to the
satisfiability of the formula and the counting constraints.

We start from a set of $l$ domains (one for every counting
constraints) and generate a set of arithmetic constraints.

\paragraph{Slice}\label{slice}

The first operation we do is what we call \emph{slicing}. It means that
the domains must be transformed so as they all have the same intervals.
In practice, we are looking for a subdivision of $[A, B)$ (where $A$
is the lower bound of every array index and $B$ the upper bound) which
can be used to express every domains. This is a simple operation that I
do not detail here, intuitively every bound of the intervals is
collected, then they are split into equality classes and ordered (in the
meantime the set of assumptions may be made bigger).

\paragraph{Partition}\label{partition}

At this point we have a set of domains who all have the same intervals
but different array constraints associated to them. For an interval
$I$, we consider every constraints associated to it, i.e.
$\phi_1, \ldots, \phi_l$. From these one we can create a set of formula
$\psi_1, \ldots, \psi_{l'}$ which are a
partition\footnote{A partition is a set of formula $\psi_1, \ldots, \psi_n$ such as $\psi_1 \lor \ldots \lor \psi_n \equiv \top$ but $\forall i, j. \; \psi_i \land \psi_j \equiv \bot$.}.
There can be at most $2^l$ formula, but of course that can be
dramatically reduced in practice with heuristics such as memoization are
inclusion detection.

\paragraph{Arithmetic Constraints}\label{arithmetic-constraints}

$I = [a, b)$ is supposed to be a finite interval (or it means that the
array is accessed on an infinite intervals, which is not supposed to
happen), so, it has a length $b - a$. As $\psi_1, \ldots, \psi_{l'}$ is a
partition, it holds that
$b - a = \sum\limits_{i = 1}^{l'} \sharp\{x \ |\ \psi_i(a_1[x], \ldots, a_n[x])\}$.We
create a new variable $v_i$ (such as $0 \le v_i$) for every
$\sharp\{x\ |\ \psi_i(a_1[x], \ldots, a_n[x]\}$ and adds the constraint
$b - a = \sum v_i$ to the solver. As long as every $\psi_i$ is
satisfiable, then we can build arrays that satisfy
$v_i = \sharp\{x\ |\ \psi_i(a_1[x], \ldots, a_n[x]\}$. Hence, we must add
$v_i > 0 \implies \psi_i(a_{i, 1}, \ldots, a_{i, n})$ (with new variables
$a_{i, 1}, \ldots$.

\paragraph{Algorithm}\label{algorithm-1}

See algorithm \ref{arrayalgo}.

\begin{algorithm}[h]
\caption{Satisfiability of arithmetic and formula with counting constraints}
\begin{algorithmic}[1]
%\Procedure{MyProcedure}{}
\State \Call{assert}{$F(\mathbf{y}, c_1, \ldots, c_n)$}
\While{$\mathcal{M} = $ \Call{check-sat}{\ } }
    \State \Call{push}{\ }
    \State $A \gets \emptyset$
    \ForAll{ $i$ in $[1..n]$}
        \State $A_i, S_i \gets $ \Call{interpret-constraint}{$c_i$, $\phi_i$,
        $\mathcal{M}$}
        \State $A \gets A \cup A_i$
        \If{$S_i$ is infinite}
            \State \Call{assert}{$\lnot \left( A \right)$}
            \State \Call{continue}{}
        \EndIf
    \EndFor
    \State \Call{slice}{$(S_i)_i$}
    \State \Call{partition}{$(S_i)_i$}
    \State \Call{assert}{$A$}
    \State \Comment{$v_\alpha$ is a variable corresponding to $phi_\alpha$, that $c_i$ selects}
    \State \Call{assert}{$\bigwedge\limits_{i=1}^n c_i = \sum\limits_{[a, b] \in S_i} \sum\limits_{\alpha} v_{\alpha}$}
    \State \Call{assert-constraints}{$(S_i)_i$}
    \If {\Call{check-sat}{\ } }
        \State \Call{pop}{\ }
        \State \Return{sat}
    \EndIf
    \State \Call{pop}{\ }
    \State \Call{assert}{$\lnot \left( A \right)$}
\EndWhile
\State \Return{unsat}
\end{algorithmic}
\label{arrayalgo}
\end{algorithm}

\paragraph{Example}\label{example-1}

TODO

\section{Arrays}\label{arrays}

The previous algorithm transforms every array reads and write into an
array constraint. This is not very efficient, because the underlying SMT
solver has no clue regarding the consistency of the values it provides
for an array read. That's why it is interesting to change this algorithm
to make it work with a usual array theory of an SMT solver. The
algorithm then has to take into account the decision that the SMT solver
when it outputs the constraints, so as the result is consistent
(i.e.~the arrays are not defined twice).

Unfortunately, as our algorithm operates outside of the solver, the
interactions they can have are quite limited. Thus, it is important to
understand how the theory of arrays is usually implemented to make it
work with the counting constraints algorithm described in the last
section.

\subsection{The theory of arrays implemented in most SMT
solvers}\label{the-theory-of-arrays-implemented-in-most-smt-solvers}

A state of the art implementation of the theory of arrays is described
in \cite{de2009generalized}. More specifically, it
explains how works the SMT solver Z3.

The only two operations on arrays are \texttt{select} and
\texttt{store}. \texttt{(select\ a\ x)} access the array $a$ at index
$x$ (also written $a[x]$), while \texttt{(store\ a\ x\ b)} creates a
new array whose elements are the same as $a$ but for $x$, where it
is set to $b$. The two axioms of this theory are:

\begin{subequations}
    \begin{align}
        \forall a:(\sigma \implies \tau), i:\sigma, v:\tau\, .\, store(a, i, v)[i] = v
        \\
        \forall a:(\sigma \implies \tau), i:\sigma, j:\sigma\, .\, i \neq j \implies store(a, i, v)[j] = a[j]
    \end{align}


    Additionnaly, there is the extensionnality axiom:

    \begin{align}
        \forall a:(\sigma \implies \tau), b:(\sigma \implies \tau)\, .\, a \neq b \iff \exists i\: a[i] \implies a[i] \neq b[i]
    \end{align}
\end{subequations}

For the first two axioms, it is clear that the SMT solver is only going
to take decisions about terms index which appear inside a \texttt{store}
and a \texttt{select}. An equality might introduce additional indexes to
make two arrays different. That's why an array extracted from a model of
an SMT solver always follows the same pattern: special values are
defined for a finite number of terms (an over-approximation of those
terms can be obtained by looking at the formula), and then a default
value. And the default value does not matter, as long as there is no
equality.

\subsection{Using an off-the-shelf SMT solver for
arrays}\label{using-an-off-the-shelf-smt-solver-for-arrays}

Thus, to ensure the consistency of the counting constraint algorithm and
the SMT solver decisions about arrays, we only have to take into account
the value at those index terms, and array equality if needed.

The first problem can be solved by tracking those index terms, then
looking at the model which slice they belong to, and group them by
slice. Then, for every array constraint $\phi_\alpha$ of this slice,
some new clauses must be sent to the solver to ensure the consistency of
this values decided by the SMT solver with the number of $x$ which
satisfies $\phi_\alpha$

The equality between two arrays must also be tracked, they must be taken
into account when the arithmetic constraints are expressed at the end of
the algorithm by adding new array constraints. Some heuristics (such as
substitution of the arrays when there is equality) can also be used.

\section{DPLL(T)}\label{dpllt}

\subsection{Introduction}\label{introduction-1}

TODO

\subsection{Integration with a general purpose DPLL based SMT
solver}\label{integration-with-a-general-purpose-dpll-based-smt-solver}

Our algorithm works by assuming a set of assumptions and checking that
cardinality values that they imply is consistent. The worst case for the
assumptions is to be a total ordering over every integer variable and an
assumptions on every atom of the formula. The number of total ordering
over $n$ variables is
$n!$\footnote{More than that when the case where two variables are equal is taken into account.},
which makes the algorithm unpractical if there is not enough constraints
on those variables in the formula.

That is why it would be interesting to have a better explanation of why
it is not \texttt{sat} and not only learn that the formula is
\texttt{unsat} with this particular ordering. That requires an
explanation from the SMT solver and a better integration with both the
arithmetic theory and the arrays theory.

Furthermore, partial assumptions are sufficient to compute the bounds of
a cardinality, and may also be sufficient to detect an inconsistency
earlier in the solving of the formula, hence avoiding useless
computations. Some work for abstract sets has already be done in this
direction \cite{cardinalityset}.

The way the algorithm works is not very different from the way a theory
is usually modeled in DPLL(T) based solver, the main differences being
that it has nested formula and has to interact with the array and
arithmetic theories. So, adapting it to be just another theory in a
solver may be worthwhile.

\newpage

\section*{Conclusion}

%related work
%optimization stuff

%In \cite{AlbertiGP16, schweikardt}, there is already a way to solve
%those formula. However, while they definitively explain that the problem
%is decidable and why, they heavily rely on quantifier elimination and
%thus may not be very practical nor easily integrated in modern SMT
%solvers.

\bibliographystyle{plain}
\bibliography{report}

\end{document}
